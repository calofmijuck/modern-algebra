\topic{Free Groups}

We learned free \textit{abelian} groups in the previous section. If a free abelian group has finite rank, then it is isomorphic to \(\Z^r\). In this section, we will drop the abelian condition.

Given a generator, the simplest group that can be built with the generator.

\section*{Words and Reduced Words}

\defn. Let \(A = \{a_i : i \in \mc{I}\}\) be a set.
\begin{enumerate}
    \item We think of \(A\) as an \textbf{alphabet}, and \(a_i \in A\) as \textbf{letters} in the alphabet.
    \item Any symbol of the form \(a_i^n\) with \(n \in \Z\) is a \textbf{syllable}.
    \item A finite string of syllables written in juxtaposition is a \textbf{word}.
    \item The \textbf{empty word} is denoted as \(1\).
\end{enumerate}

Suppose we are given a word \(a_i^m a_i^n\). Then we can \textit{simplify} it as \(a_{i}^{m+n}\). We call this \textbf{contraction}.

\defn. A \textbf{reduced word} is a word with no possible contractions. So \(a_{i_j} \neq a_{i_{j+1}}\) for all \(j\).

\smallskip

We take the following as granted.

\textbf{Claim.} Any word can be reduced into a unique reduced word.

\section*{Free Group}

Let \(A\) be a set and let \(F[A]\) be a set of reduced words. We define multiplication on \(F[A]\) by juxtaposition and applying contraction to get a new reduced word.\footnote{We must check that \((F[A], \cdot)\) is a group, but it is obvious...}

\ex. \(a_1^2 a_2^{-3} a_3^2 \cdot a_3^{-2} a_2^2 a_4 = a_1^2 a_2\inv a_4\).

\defn. \note{Free Group}
\begin{enumerate}
    \item We define \(F[A]\) as the \textbf{free group generated by \(A\)}.
    \item Given a group \(G\), suppose that \(G = \span{A}\) for a subset \(A \subset G\). If \(G \simeq F[A]\), then we say that \(G\) is \textbf{free} on \(A\).
    \item If a group is free on some non-empty set \(A\), then it is called a \textbf{free group}.
\end{enumerate}

\ex. Let \(A = \{a\}\). Then \(F[A] = \{a^n : n \in \Z\}\) and thus \(F[A] \simeq \Z\). This is a special case!

\prop. If \(F[A]\) is abelian, then \(\abs{A} = 1\).

\pf If \(a, b \in A\) with \(a \neq b\), then \(ab \neq ba\) as reduced words. \qed

The proofs for the following two theorems are omitted. Check if you are curious.

\thm. \(F[A] \simeq F[B]\) if and only if \(\abs{A} = \abs{B}\). Hence we can define the \textbf{rank} of \(F[A]\).

\thm. Nontrivial proper subgroup of a free group is free.

Recall that for free \textit{abelian} groups, if \(G \simeq \Z^n\) and \(H \leq G\), then \(H\) is a free abelian group with rank \(s \leq n\). So the rank of a free \textit{abelian} group was sort of a measure for the size of the group. However, in free groups, that does not work! The rank does not give us much information.

\ex. \(F[\{x, y\}]\) has rank \(2\). Consider \(F[\{x^k y x^{-k} : k \in \N \}]\), which is strictly smaller than \(F[\{x, y\}]\) but has infinite rank.

\section*{Homomorphisms of Free Groups}

This part is the most important because of the following reasons: It tells us how we understand general groups using free groups. It also tells us what \textit{free} means.

\thm. Let \(G\) be a group generated by \(A = \{a_i : i \in \mc{I}\}\). Let \(G'\) be a group \(\{a_i' : i \in \mc{I}\} \subset G'\).
\begin{enumerate}
    \item Then there exists \textit{at most} one homomorphism \(\varphi : G \ra G'\) such that \(a_i \mapsto a_i'\).
    \item If \(G\) is free on \(A\), then such homormophism is unique.
\end{enumerate}

\pf A homomorphism \(\varphi\) is completely determined by its values on the generating elements. Thus there can only be at most one homomorphism. If \(G = F[A]\), then the map is well-defined, since \(F[A]\) consists of reduced words. No two different formal products in \(F[A]\) are equal. \qed

The meaning of \textit{at most}: we cannot guarantee the well-definedness of \(\varphi\).

This theorem explains the meaning of \textit{free}. Suppose that \(G = F[\{a, b\}] \quotient \span{ab - ba}\). If \(a \mapsto a'\), \(b \mapsto b'\), then \(ab-ba \mapsto a'b' - b'a'\). While \(ab - ba = 0 \in G\), but \(a'b' - b'a' \in G'\) may not be \(0\) since we have no information about \(a', b'\). If we remove the relation \(ab - ba = 0\), then well-definedness automatically follows since arbitrary reduced words are not \(0\).

In advanced algebra class, we define free groups as groups that satisfy (2). So they have the least relations, and we can define many kinds of homomorphisms on \(G\). Then we prove that to satisfy such a property, free groups must a a set of reduced words.

We will use this in the next section when we learn group presentations.

\thm. Every group is a homomorphic image of a free group.

\pf Suppose \(G = \span{A}\) where \(A = \{a_i : i \in \mc{I}\} \subset G\). Define \(\varphi : F[A] \ra G\) be the unique homomorphism such that \(a_i \ra a_i\). Then \(\varphi\) is onto, so \(\im \varphi = G \simeq F[A] \quotient \ker \varphi\). \qed

We call \(A\) a \textbf{generator}, and \(\ker \varphi\) a \textbf{relation}.

\pagebreak

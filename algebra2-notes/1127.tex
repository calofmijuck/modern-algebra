\defn. \note{Galois Group} If \(K\) is a finite normal extension of \(F\), then \(G(K/F)\) is called the \textbf{Galois group} of \(K\) over \(F\).

\thm. \note{Galois} Let \(K\) be a finite normal extension of \(F\) and let \(E\) be an intermediate field between \(F\) and \(K\).
\begin{enumerate}
    \item \(E = K_{G(K/E)}\).
    \item If \(H \leq G(K/F)\), then \(G(K/K_H) = H\).
    \item \([K : E] = \abs{G(K/E)}\), \([E : F] = \bigl(G(K/F) : G(K/E)\bigr)\).\footnote{\(E\) may not be a normal extension of \(F\), so we cannot write \(\abs{G(E/F)}\) here.}
    \item \(E\) is a normal extension of \(F\) if and only if \(G(K/E) \nsub G(K/F)\). In this case,
          \[
              G(E/F) \simeq \frac{G(K/F)}{G(K/E)}.
          \]
    \item The subgroup diagram of \(G(K/F)\) is the inverted diagram of intermediate fields of \(K\) over \(F\). i.e, this is an \textit{order-reversing} bijection.
\end{enumerate}

\pf \note{1} By definition, \(E \leq K_{G(K/E)}\). Now let \(\alpha \in K \bs E\). Since \(K\) is a finite (algebraic) extension of \(E\), \(\alpha\) must have a conjugate \(\beta \in K \bs E\). Then the conjugation isomorphism \(\psi_{\alpha, \beta} : E(\alpha) \ra E(\beta)\) can be extended to an automorphism of \(K\) that fixes \(E\). Then \(\alpha \in K \bs E\) is not fixed, so \(\alpha \notin K_{G(K/E)}\). Thus \(K_{G(K/E)} \leq E\).

% Let \(K = E(\alpha_1, \dots, \alpha_n)\) and \(\deg(\alpha_i, E) \geq 2\). For \(\alpha_i\), there exists a conjugate \(\beta_i \in K\). Consider the conjugation isomorphism \(\psi_{\alpha, \beta}\) which can be extended to an automorphism of \(K\), which implies that \(\psi_{\alpha, \beta} \in G(K/E)\). There exists an isomorphism that does not fix \(\alpha_i \notin E\). Thus \(K_{G(K/E)} \leq E\).

\note{2} First, take \(\sigma \in H\). For \(k \in K_H\), \(\sigma(k) = k\) by definition of \(K_H\). Then \(\sigma \in G(K/K_H)\) and \(H \leq G(K/K_H)\). For the other direction, note that \(H \leq G(K/K_H) \leq G(K/F)\). So we assume \(H < G(K/K_H)\) and derive a contradiction.

Then \(\abs{H} < \abs{G(K/K_H)} = [K : K_H]\), since \(K\) is a finite normal extension of \(K_H\). Also, by the primitive element theorem, choose \(\alpha \in K\) such that \(K = K_H(\alpha)\). Now consider the polynomial \(f(x) = \prod_{\sigma_i \in H}\bigl(x - \sigma_i(\alpha)\bigr)\), which has degree \(\abs{H}\). This polynomial has coefficients in \(K_H\), since for every \(\sigma \in H\), \(\tilde{\sigma}\bigl(f(x)\bigr) = \prod_{\sigma_i \in H}\bigl(x - \sigma(\sigma_i(\alpha))\bigr) = f(x)\). Also, one of \(\sigma_i\) is the identity, so \(f(\alpha) = 0\). Thus,
\[
    \deg(\alpha, K_H) \leq \abs{H} < \abs{G(K/K_H)} = [K_H(\alpha) : K_H] = \deg(\alpha, K_H),
\]
arriving at a contradiction.

\note{3} \(K\) is a finite normal extension of both \(E\) and \(F\). Thus \([K : E] = \abs{G(K/E)}\) and \([K : F] = \abs{G(K/F)}\), which implies
\[
    [E : F] = \frac{[K : F]}{[K : E]} = \frac{\abs{G(K/F)}}{\abs{G(K/E)}} = \bigl(G(K/F) : G(K/E)\bigr).
\]

\note{4} \(E\) is a normal extension of \(F\).
\begin{itemize}
    \item[] \hspace{-20pt} \(\iff\) \(E\) is a splitting field over \(F\).

        \(E\) is separable over \(F\) by assumption.

    \item[] \hspace{-20pt} \(\iff\) every isomorphism of \(E\) onto a subfield of \(\bar{F}\) leaving \(F\) fixed induces an automorphism of \(E\).

        These isomorphisms can be extended to an automorphism of \(K\), since \(K\) is normal over \(F\). Thus \(\sigma \in G(K/F)\) induces those isomorphisms.

    \item[] \hspace{-20pt} \(\iff\) \(\sigma(\alpha) \in E\) for any \(\sigma \in G(K/F)\) and \(\alpha \in E\).

    \item[] \hspace{-20pt} \(\iff\) \(\tau(\sigma(\alpha)) = \sigma(\alpha)\) for all \(\tau \in G(K/E)\).

        The reverse direction uses that \(E = K_{G(K/E)}\).

    \item[] \hspace{-20pt} \(\iff\) \(\sigma\inv\tau\sigma|_E = id\) for any \(\tau \in G(K/E)\) and \(\sigma \in G(K/F)\).

    \item[] \hspace{-20pt} \(\iff \sigma\inv\tau\sigma \in G(K/E)\), showing that \(G(K/E) \nsub G(K/F)\).
\end{itemize}

If \(E\) is normal over \(F\), \(E\) is a splitting field over \(F\). So \(\sigma \in G(K/F)\) induces an automorphism \(\sigma|_E\) of \(E\), and \(\sigma |_E \in G(E/F)\). The map \(\varphi : G(K/F) \ra G(E/F)\) defined as \(\sigma \mapsto \sigma|_E\) is clearly a well-defined homomorphism. By the isomorphism extension theorem, it is surjective, and \(\ker\varphi = G(K/E)\). The result follows by the first isomorphism theorem.

\note{5} \(F \leq E \imp G(K/E) \leq G(K/F)\). The bijection reverses the order. \qed

From the above theorem, we see that the mapping \(H \mapsto K_H\) and \(E \mapsto G(K/E)\) are inverses.
\[
    \begin{tikzcd}[row sep=tiny, every arrow/.append style={shift left}]
        H = G(K/K_H) \arrow{rr}{\varphi} && K_H \arrow{ll}{\psi} \\
        G(K/E) \arrow{rr}{\varphi} && E = K_{G(K/E)} \arrow{ll}{\psi}
    \end{tikzcd}
\]

\smallskip

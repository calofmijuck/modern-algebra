\medskip

We will now see examples where Sylow theorems can be used.

\ex. Let \(\abs{G} = p^r\) (\(r \geq 2\)). (finite \(p\)-group) By the first Sylow theorem, there exists a normal subgroup of order \(p^{r-1}\). Thus \(G\) is not simple. \qed

\ex. Let \(\abs{G} = 15\). Since \(15 = 3 \times 5\), \(G\) is not simple and \(G \simeq \Z_{3\times 5}\) by the above theorem. \qed

\ex. Let \(\abs{G} = 20\). Note that \(20 = 2^2 \times 5\).
\begin{itemize}
    \item The number of Sylow \(2\)-subgroups can be either 1 or 5 by the third Sylow theorem.
    \item The number of Sylow \(5\)-subgroups should be 1 also by the third Sylow theorem.
\end{itemize}
Hence the Sylow \(5\)-subgroup of \(G\) is unique, so it is normal and \(G\) is not simple. \qed

We always check the number of Sylow \(p\)-subgroups first.

\ex. Let \(\abs{G} = 30\). Note that \(30 = 2 \times 3 \times 5\). By the third Sylow theorem,
\begin{itemize}
    \item The number of Sylow \(2\)-subgroups is either 1, 3, 5 or 15.
    \item The number of Sylow \(3\)-subgroups is either 1 or 10.
    \item The number of Sylow \(5\)-subgroups is either 1 or 6.
\end{itemize}
Our situation is not as simple as last time. We have to show that the number of Sylow \(p\)-subgroups is \(1\) for some \(p\). So suppose that the number of Sylow \(3\)-subgroups is \(10\) and the number of Sylow \(5\)-subgroups is \(6\). Let \(P_1, \dots, P_{10}\) be distinct Sylow \(3\)-subgroups and \(Q_1, \dots, Q_6\) be distinct Sylow \(5\)-subgroup of \(G\).

We know that \(\abs{P_i} = 3\) and \(\abs{Q_i} = 5\). Our claim is that \(P_i \cap P_j = \{e\}\) for \(i \neq j\). This is because a group of order \(3\) is isomorphic to \(\Z_3\) and is generated by any element other than \(e\). Using a similar argument for \(Q_i\), we have
\[
    \abs{\bigcup_{i=1}^{10} P_i} = 1 + 2 \times 10 = 21, \quad \abs{\bigcup_{i=1}^6 Q_i} = 1 + 4 \times 6 = 25.
\]
Thus
\[
    \abs{\bigcup_{i=1}^{10} P_i \cup \bigcup_{i=1}^6 Q_i} = \overbrace{1}^{\text{identity}} + \overbrace{20}^{\text{order 3 elements}} + \overbrace{24}^{\text{order 5 elements}} = 45.
\]
But \(\abs{G} = 30\), contradiction. Thus either the number of Sylow \(3\)-subgroups or \(5\)-subgroups must be \(1\), therefore \(G\) is not simple. \qed

\ex. Let \(\abs{G} = 48\). Note that \(48 = 2^4 \times 3\). If there is a unique Sylow \(2\)-subgroup, we are done, so assume not. Let \(H, K\) be distinct Sylow \(2\)-subgroups. By the lemma above,
\[
    \abs{HK} = \frac{\abs{H} \abs{K}}{\abs{H \cap K}} = \frac{16 \times 16}{\abs{H \cap K}} \leq \abs{G} = 48.
\]
Thus \(\abs{H \cap K}\) should be at least \(6\), and should divide \(16\), since \(H \cap K\) is a subgroup of \(H\) and \(K\). Then the only possibility is \(\abs{H \cap K} = 8\). Then \(\ind{H}{H \cap K} = \ind{K}{H \cap K} = 2\), so \(H \cap K \nsub H, K\).

Now we consider \(N[H \cap K]\).\footnote{We actually want this group to be equal to \(G\)...}
\begin{itemize}
    \item By the above result, \(H, K \leq N[H \cap K]\). Thus \(16 \mid \abs{N[H \cap K]}\).
    \item \(N[H \cap K] > 16\), since it contains two distinct groups \(H\) and \(K\) of order \(16\).
    \item \(N[H \cap K] \leq G\), so \(\abs{N[H\cap K]} \mid 48\).
\end{itemize}

Overall, \(\abs{N[H\cap K]} = 48\) and \(N[H \cap K] = G\), so \(H \cap K \nsub G\) and \(G\) is not simple. \qed

\ex. Let \(\abs{G} = 3 \times 5 \times 17\). \(G\) is abelian and cyclic.

\pf We only have to show that \(G\) is abelian, because if \(G\) is abelian, \(G\) is cyclic by the fundamental theorem of finitely generated abelian groups. We will show that the commutator subgroup \(C(G)\) is trivial.

By the third Sylow theorem, \(G\) has a unique Sylow \(17\)-subgroup \(H\), and \(H \nsub G\). We consider \(G/H\), which has \(15\) elements. In fact, \(G/H \simeq \Z_{15}\) by the above example, thus it is abelian and cyclic. Recall that \(C(G) \leq N\) if and only if \(G/N\) is abelian. So \(C(G) \leq H\), so \(C(G)\) must have order \(1\) or \(17\).

Again by the third Sylow theorem,
\begin{itemize}
    \item The number of Sylow \(3\)-subgroups is either \(1\) or \(85\).
    \item The number of Sylow \(5\)-subgroups is either \(1\) or \(51\).
\end{itemize}
Using a similar argument as above, we cannot have \(85\) distinct Sylow \(3\)-subgroups and \(51\) distinct Sylow \(5\)-subgroups at the same time. Check that \(1 + 2 \times 85 + 4 \times 51 = 375 > \abs{G}\). So either there is a unique Sylow \(3\)-subgroup, or a unique Sylow \(5\)-subgroup.
\begin{itemize}
    \item If \(G\) has a unique Sylow \(3\)-subgroup \(H_3\). Then it is normal, so \(G/H_3 \simeq \Z_{85}\) and is abelian. Then \(C(G) \leq H_3\), so \(C(G)\) must have order \(1\) or \(3\).

    \item If \(G\) has a unique Sylow \(5\)-subgroup \(H_5\). Similarly, \(G/H_5 \simeq \Z_{51}\), so \(C(G) \leq H_5\) and \(C(G)\) must have order \(1\) or \(5\).
\end{itemize}
Since \(C(G)\) must have order \(1\) or \(17\), we see that \(C(G)\) must be \(\{e\}\) for both cases. \qed

Sylow theorems will be used again when we study more field theory. We will not use them to classify finite groups, but the ideas and techniques in the proofs will be used again.

\pagebreak

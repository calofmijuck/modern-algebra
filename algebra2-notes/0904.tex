\setcounter{chapter}{5}

\chapter{Extension Fields}

Topics: Extension Field, Galois Theory, Advanced Group Theory.

\setcounter{topic}{28}
\topic{Introduction to Extension Fields}

Examples of fields: \(\Z_p\), \(\Q\), \(\R\), \(\C\) and the field of quotients. We now introduce more examples. The idea is the following:

\recall Let \(R\) be a commutative ring with unity.\footnote{In this semester, \(R\) will almost always have a unity.} If \(M\) is a maximal ideal of \(R\), then \(R/M\) is a field.

We will construct new examples from this proposition. For example, given a polynomial ring \(F[x]\) where \(F\) is a field, \(F[x]/M\) is a field if \(M\) is a maximal ideal of \(F[x]\). Our task will be to describe \(M\), and then \(F[x]/M\).

\recall \(M\) is a maximal ideal of \(F[x] \iff M = \span{p(x)}\), with \(p(x)\) irreducible. So \(F[x]\) is a principal ideal domain, i.e, all ideals of \(F[x]\) are principal (generated by a single element).

Thus if we set \(M = \span{p(x)}\) where \(p(x)\) is a irreducible polynomial, \(F[x]/M\) would be a field.

\question Why do we call \(F[x]/M\) an \textit{extension} field? Because \(F \hookrightarrow {F[x]} \quotient {\span{p(x)}}\).\footnote{\(a \in F[x]\), injective.}

\question Why do we consider extension fields?

Consider \(x^2 + 1 \in \R[x]\). This polynomial has no solutions in \(\R\), but we want a new field containing \(R[x]\) and also the \textit{roots} of \(x^2 + 1\). So we consider \(\R[x] \quotient \span{x^2 + 1}\).

\pagebreak

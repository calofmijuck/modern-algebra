\cor. If \(D\) is a UFD and \(F = Q_D\), then a non-constant \(f(x) \in D[x]\) factors into a product of two polynomials of lower degrees \(r\) and \(s\) in \(F[x]\) if and only if it has a factorization into polynomials of the same degrees \(r\) and \(s\) in \(D[x]\).

\pf See (1) in the above proof, the converse holds because \(D[x] \subset F[x]\). \qed

\rmk If \(f(x) \in D[x]\) is primitive, then \(f(x) \in D[x]\) is irreducible if and only if \(f(x) \in Q_D[x]\) is irreducible.

\thm. If \(D\) is a UFD, then \(D[x]\) is also a UFD.

\pf Take \(f(x) \in D[x]\), with \(\deg f \geq 1\). If \(\deg f = 0\), then we are done since \(D\) is a UFD. Write \(f(x) = g_1(x) g_2(x) \cdots g_r(x) \in D[x]\), where \(r\) is maximal and \(\deg g_i \geq 1\). Then \(g_i(x) = c_ih_i(x)\) where \(c_i \in D\) and \(h_i(x)\) are primitive and irreducible by the maximality of \(r\). Then we have \(f(x) = (c_1 \cdots c_r)h_1(x) \cdots h_r(x)\). We can factorize \(c_1 \cdots c_r\) into \(d_1 \cdots d_s\) where \(d_i\) are irreducible in \(D\).

For uniqueness, let \(F = Q_D\). We will use the fact that \(F[x]\) is a UFD. First factorize \(f(x)\) in \(D[x]\) and get \(f(x) = g_1(x) \cdots g_s(x)\) where \(g_i(x)\) are irreducible factors in \(D[x]\). (the factors are also primitive) This factorization can be viewed as a factorization in \(F[x]\), where \(g_i(x)\) are irreducible in \(F[x]\). This factorization is unique in \(F[x]\) up to unit factors, so it guarantees the uniqueness in \(D[x]\).\footnote{Check if this is right...} \qed

\cor. \(F[x_1, x_2, \dots, x_n]\) is a UFD.

\rmk \(F[x, y]\) is a UFD, but not a PID since \(\span{x, y} \subset F[x, y]\) is not principal.

\pagebreak

\topic{Euclidean Domains}

We could do Euclidean division in the rings \(\Z\) and \(F[x]\). We want to study if this property can be extended to other domains.

\defn. \note{Euclidean Domain} Let \(D\) be an integral domain having a \textbf{Euclidean norm} \(\nu : D \ra \Z_{\geq 0}\) that satisfies the following.
\begin{enumerate}
    \item For any \(a \in D\) and \(b \in D \bs \{0\}\), there exists \(q, r \in D\) such that
    \begin{center}
        \(a = bq + r\) \quad where \quad \(\nu(r) < \nu(b)\) or \(r = 0\).
    \end{center}
    \item For any \(a, b \in D \bs \{0\}\), \(\nu(a) \leq \nu(ab)\).
\end{enumerate}
Then \(D\) is called a \textbf{Euclidean domain}.

The Euclidean norm gives a \textit{ordering} of elements on \(D\).

\ex. For \(\Z\), \(\nu\) is the mapping \(a \mapsto \abs{a}\). For \(F[x]\), \(f(x) \mapsto \deg f\).

\thm. Every Euclidean domain is a PID.

\pf Suppose \((D, \nu)\) is a Euclidean domain, let \(N \subset D\) be an nontrivial ideal. Take an element \(n \in N\) having the minimal \(\nu(n)\). Now we show that \(N = \span{n}\).

Suppose not, and take \(a \in N \bs \span{n}\). Then there exists \(q, r \in D\) such that \(a = nq + r\) where \(\nu(r) < \nu(n)\) or \(r = 0\). But \(r \neq 0\) since \(a \notin \span{n}\). \(r = a - nq \in \span{n}\) and this is not possible. \qed

\thm. Let \((D, \nu)\) be a Euclidean domain.
\begin{enumerate}
    \item \(\nu(1)\) is minimal among \(\nu(a)\) for all \(a \in D \bs \{0\}\).
    \item \(u \in D\cross \iff \nu(u) = \nu(1)\).
\end{enumerate}

\pf \note{1} \(\nu(1) \leq \nu(1 \cdot a) \leq \nu(a)\).

\note{2} \note{\mimp} \(\nu(u) \leq \nu(u \cdot u \inv) = \nu(1)\). Since \(\nu(1)\) is minimal, \(\nu(u) = \nu(1)\).

\note{\mimpd} If \(\nu(u) = \nu(1)\), there exists \(q, r \in D\) such that \(1 = uq + r\) where \(r = 0\) or \(\nu(r) < \nu(u)\). But \(\nu(r) < \nu(u) = \nu(1)\), so \(r = 0\) and \(u\) is a unit. \qed

\thm. Let \((D, \nu)\) be a Euclidean domain. For \(a, b \in D \bs \{0\}\), there exists \(q_1, r_1 \in D\) such that \(a = bq_1 + r_1\) where \(r_1 = 0\) or \(\nu(r_1) < \nu(b)\). Let \(r_0 = b\), and continue the process. We can write \(r_{i-1} = r_i q_{i+1} + r_{i+1}\) for some \(r_{i+1}\) such that \(r_{i+1} = 0\) or \(\nu(r_{i+1}) < \nu(r_{i})\). Stop when \(r_s = 0\). Then \(r_{s-1}\) is a \(\gcd\) of \(a, b\). (\(r_0 = b\))

\cor. \(d = r_{s-1} = \gcd(a, b)\) can be expressed as \(d = \lambda a + \mu b\) for some \(\lambda, \mu \in D\).

\pf Read in textbook. \qed

We learned that using abstract languages, Euclidean division, factorization and principal ideals are not equivalent properties.

\pagebreak

\topic{Gaussian Integers and Multiplicative Norms}

Are there examples other than \(\Z\) and \(F[x]\)? We learn a new example in this section.

\defn. \note{Gaussian Integers} \(\Z[i] = \{a + bi \in \C : a, b \in \Z\}\). The norm \(N : \Z[i] \ra \Z_{\geq 0}\) is defined as \(a + bi \mapsto a^2 + b^2\).

\lemma. For \(N : \Z[i] \ra \Z_{\geq 0}\) and \(\alpha, \beta \in \Z[i]\), the following hold.
\begin{enumerate}
    \item \(N(\alpha) \geq 0\).
    \item \(N(\alpha) = 0 \iff \alpha = 0\).
    \item \(N(\alpha\beta) = N(\alpha)N(\beta)\).\footnote{Proof is easy if we use polar forms of complex numbers.}
    \item \(\Z[i]\) is an integral domain, as a subset of the complex field \(\C\).
\end{enumerate}

\pf Easy. \qed

\smallskip

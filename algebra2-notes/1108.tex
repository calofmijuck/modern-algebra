\ex. Consider
\[
    G = G(\Q(\sqrt{2}, \sqrt{3}) \quotient \Q) = \{id, \psi_{\sqrt{2}, -\sqrt{2}}, \psi_{\sqrt{3}, -\sqrt{3}}, \psi_{\sqrt{3}, -\sqrt{3}} \circ \psi_{\sqrt{2}, -\sqrt{2}}\}.
\]
In this case, \(\Q\) is the fixed field of \(G\). Consider the basis \(\Q \oplus \sqrt{2} \Q \oplus \sqrt{3} \Q \oplus \sqrt{6}\Q\).

\section*{The Frobenius Automorphism}

We will use this when we study more about finite fields.

\thm. \note{Frobenius Automorphism} Let \(F\) be a finite field of characteristic \(p\). The map
\[
    \sigma_p : F \ra F, \quad a \mapsto a^p \quad (a \in F)
\]
is called the \textbf{Frobenius automorphism}. We have \(F_{\sigma_p} \simeq \Z_p\).

\pf \(\sigma_p\) is an automorphism by freshman's dream. Consider an embedding \(\Z_p \hookrightarrow F\). For \(n \in \Z_p\), \(\sigma_p(n) = n^p = n\) by Fermat's little theorem. Thus \(n \in F_{\sigma_p}\). Also, \(F\) is a field, so the polynomial \(x^p - x\) has at most \(p\) zeros. Elements of \(\Z_p\) are exactly the \(p\) zeros, so \(F_{\sigma_p} = \Z_p\). \qed

\topic{The Isomorphism Extension Theorem}

In the last section, we learned that if \(\irr(\alpha, F) = \irr(\beta, F)\) for two algebraic elements \(\alpha, \beta\), then \(F(\alpha), F(\beta)\) have the same algebraic structure. But where does \(\alpha\) and \(\beta\) live in? There can be many ways to algebraically extend the field \(F\), but using the \textit{isomorphism extension theorem}, we will learn that the algebraic closure of \(F\) is unique up to isomorphism.

We also want to study the existence and the number of automorphisms of a field.

\thm. \note{Isomorphism Extension Theorem} Let \(E\) be an algebraic extension field of \(F\), and let \(\sigma : F \ra F'\) be a field isomorphism. Then \(\sigma\) can be extended to an isomorphism \(\tau : E \ra E' \leq \bar{F'}\) such that \(\tau\mid_F = \sigma\).
\[
    \begin{tikzcd}
        \bar{F} \arrow[dash]{d} & &\bar{F'} \arrow[dash]{d} \\
        E \arrow[dash]{d} \arrow{rr}{\tau}[swap]{\simeq} & & E' \arrow[dash]{d} \\
        F \arrow{rr}{\sigma}[swap]{\simeq} & & F'
    \end{tikzcd}
\]
Given \(F, F', E, \sigma\), the theorem tells us that an extension \(\tau : E \ra E' \leq \bar{F'}\) of \(\sigma\) exists.

\pagebreak

\cor. Let \(E \leq \bar{F}\) be an algebraic extension of \(F\). If \(\alpha, \beta\) are conjugate over \(F\), the conjugation isomorphism \(\psi_{\alpha, \beta} : F(\alpha) \ra F(\beta)\) can be extended to an isomorphism \(\tau : E \ra E' \leq \bar{F}\).
\[
    \begin{tikzcd}
        E \arrow[dash]{d}{\text{alg.}} \arrow{rr}{\tau}[swap]{\simeq} & & E' \arrow[dash]{d} \\
        F(\alpha) \arrow{rr}{\psi_{\alpha, \beta}}[swap]{\simeq} & & F(\beta)
    \end{tikzcd}
\]

\cor. Let \(\bar{F}\) and \(\bar{F'}\) be algeraic closures of \(F\). Then \(\bar{F} \overset{\varphi}{\simeq} \bar{F'}\) with \(\varphi\) fixing \(F\).

\pf By the extension theorem, extend the identity isomorphism \(id : F \ra F\), to \(\tau : \bar{F} \ra E \leq \bar{F'}\). \(\tau\) will leave \(F\) fixed. Next, we show that \(\tau\) is onto \(\bar{F'}\). Consider the isomorphism \(\tau\inv : E \ra \bar{F}\), extend it to an isomorphism \(\varphi : \bar{F'} \ra E' \leq \bar{F}\). Since \(\tau\inv\) is already onto \(\bar{F}\), \(E' = \bar{F}\) and \(\varphi : \bar{F'} \ra \bar{F}\) is the desired isomorphism fixing \(F\). \qed
\[
    \begin{tikzcd}
        & & \bar{F'} \arrow[dash]{d} \arrow{rr}{\varphi}[swap]{\simeq} & & E' \arrow[equal]{d}\\
        \bar{F} \arrow[dash]{d} \arrow{rr}{\tau}[swap]{\simeq} & & E \arrow[dash]{d} \arrow{rr}{\tau\inv}[swap]{\simeq} & & \bar{F} \\
        F \arrow[dash]{rr}{id}[swap]{\simeq} & & F
    \end{tikzcd}
\]

\section*{Index of a Field Extension}

Now that existence is given by the isomorphism extension theorem, we want to count how many there are. In particular, we want to find isomorphisms from \(E\) to a subfield of \(\bar{F}\) that leaves \(F\) fixed.

\defn. \note{Index} Let \(E\) be a finite extension of \(F\). The number of isomorphisms of \(E\) onto a subfield of \(\bar{F}\) leaving \(F\) fixed is called the \textbf{index} of \(E\) over \(F\), denoted as \(\{E : F\}\).

\ex. Isomorphisms of \(\Q(\sqrt{2})\) that fixes \(\Q\) must map \(\sqrt{2}\) to itself or \(-\sqrt{2}\). Therefore, \(\{\Q(\sqrt{2}) : \Q\} = 2\).

Using this definition, we want to find the number of automorphisms between the roots of some polynomial, that was used to extend \(F\) to \(E\). The following theorem shows the well-definedness of the index.

\thm. Let \(E\) be a finite extension field of \(F\), and let \(\sigma : F \ra F'\) be an isomorphism. The number of extensions \(\tau : E \ra E' \leq \bar{F'}\) is finite. Moreover, the number of extensions is completely determined by \(E\) and \(F\).

\pf We have to show that the number of extensions is independent of \(F'\) and \(\sigma\). We already know the existence of \(\tau\) and \(E'\).

Given two isomorphisms \(\sigma_1 : F \ra F_1'\) and \(\sigma_2 : F \ra F_2'\), \(\sigma_2 \sigma_1\inv : F_1' \ra F_2'\) is an isomorphism. We can extend this into an isomorphism \(\lambda : \bar{F_1'} \ra \bar{F_2'}\). Now for each \(\tau_1 : E \ra \bar{F_1'}\) that extend \(\sigma_1\), we can choose \(\tau_2 = \lambda \circ \tau_1\). Similarly, we can also extend \(\sigma_2\) to get \(\tau_2 : E \ra \bar{F_2'}\) and choose \(\tau_1 : \lambda\inv \circ \tau_2\). Thus the number of extensions \(\tau\) of \(\sigma\) must be the same, independent of \(F_1'\) and \(\sigma\).

% If \(\tau_1 \neq \tau_1'\) extensions of \(\sigma_1\), then \(\lambda \circ \tau_1 \neq \lambda \circ \tau_1'\). Thus the number of \(\tau_2\) is greater than equal to the number of \(\tau_1\). The other way can be shown similarly. Thus the numbers are equal.

\[
    \begin{tikzcd}
        \bar{F_1'} \arrow[dash]{d} \arrow{rrrr}{\lambda} && && \bar{F_2'} \arrow[dash]{d} \\
        \tau_1(E) \arrow[dash]{d}&& E \arrow{ll}{\tau_1} \arrow{rr}{\tau_2} \arrow[dash]{d} && \tau_2(E) \arrow[dash]{d} \\
        F_1' && F \arrow{ll}{\sigma_1} \arrow{rr}{\sigma_2} && F_2'
    \end{tikzcd}
\]

The number is also finite since \(E = F(\alpha_1, \dots, \alpha_n)\) for some \(\alpha_i \in E\). Consider the \(\irr(\alpha_i, F) = a_0 + a_1 x + \cdots + a_{n_i} x^{n_i}\). Since \(\tau\) fixes \(F\), \(\tau(\alpha_i)\) also be a zero of \(\tau(a_0) + \tau(a_1) x + \cdots + \tau(a_{n_i}) x^{n_i}\). Thus there are finitely many choices for \(\tau(\alpha_i)\) and the total number must be less than the product of the number of zeros of each irreducible polynomial of \(\alpha_i\). \qed

\topic{Group Presentations}

Recall that for finite abelian groups, we classified them, but for finite \textit{nonabelian} groups, we don't really know much. We learned that composition series is unique, Sylow theorems tell us whether a group is simple or not. This is all we know...

\ex. Let \(F[A]\) be a free group on \(A\), and let \(C\) be a commutator subgroup of \(F[A]\). Then \(C \nsub F[A]\), and \(F[A] \quotient C\) is a free abelian group of basis \(\bar{A}\).

\defn. Let \(A\) be a set and \(\{r_i\} \subset F[A]\). Let \(R\) be the smallest normal subgroup of \(F[A]\) generated by \(\{r_i\}\).
\begin{enumerate}
    \item If \(G \simeq F[A] \quotient R\), then it is called the \textbf{group presentation} of \(G\).\footnote{\(A\) is a generator, \(R\) is a relator.}
    \item If \(A\) and \(\{r_i\}\) are both finite then the presentation is called \textbf{finite presentation}.
    \item Let \(A = \{a_i : i \in \mc{I}\}\). \textbf{Group presentation of \(G\)} is denoted as
          \begin{center}
              \(\span{a_i \mid r_j}_{i \in \mc{I},\, j \in \mc{J}}\) \quad or \quad \(\span{a_i \mid r_j = 1}_{i \in \mc{I},\, j \in \mc{J}}\).
          \end{center}
\end{enumerate}

\section*{Isomorphic Presentations}

\ex. \(\abs{G} = 6\).
\begin{enumerate}
    \item If \(G\) is abelian, then \(G \simeq \Z_6 \simeq \span{a \mid a^6 = 1}\).
    \item If \(G\) is not abelian, by the Sylow theorem, there exists \(a, b \in G\) such that \(a^2 = 1, b^3 = 1\). Then \(G \simeq S_6 \simeq \span{a, b \mid a^2 = b^3 = 1, ab = b^2a}\).
\end{enumerate}

\rmk Group presentation is not unique. We can have \textbf{isomorphic presentations} where two different presentations of a group can give isomorphic groups. For example, suppose that \(G = \span{a, b}\), \(a^2 = b^3 = 1\) and \(ab = ba\). Then by enumerating elements we see that \(G \simeq \Z_6 \simeq \span{a, b \mid a^2 = b^3 = 1, ab = ba}\).

\section*{Applications of Group Presentations}

We classify small groups using group presentations.

\ex. \(\abs{G} = 10\).
\begin{enumerate}
    \item If \(G\) is abelian, then \(G \simeq \Z_{10}\).
    \item If \(G\) is not abelian, by the Sylow theorem, there exists a normal subgroup of order \(5\). Let it be \(H\). Then \(\abs{H} = 5\), so it is cyclic and there is a generator \(a \in H\) such that \(\abs{a} = 5\). Take \(b \in G \bs H\), then \(b^2 \in H\) since \(G \quotient H\) has order \(2\). Our claim is that \(b^2 = 1\). If not, all elements of \(H\) except \(1\) has order \(5\), so \(\abs{b} = 10\). Then \(b\) is a generator of \(G\) and \(G\) is cyclic, contradicting that \(G\) is not abelian.

          Thus \(G = \span{a, b}\) and \(a^5 = 1, b^2 = 1\). To determine the structure of \(G\), we have to find what \(ba\) is. We see that \(ba \neq a^k, b, ab\). So \(ba\) is either \(a^2b, a^3b, a^4b\).

          \begin{itemize}
              \item If \(ba = a^2b\), then \(a = b^2a = ba^2b = a^4b^2 = a^4\). Thus \(a^3 = 1\), contradiction.
              \item If \(ba = a^3b\), then \(a = b^2a = ba^3b = a^9 b^2 = a^9\). Thus \(a^8 = a^3 = 1\), contradiction.
              \item If \(ba = a^4b\), the dihedral group of order 10 satisfies this.
          \end{itemize}
          Thus, \(G \simeq D_5 \simeq \span{a, b \mid a^5 = b^2 = 1, ba = a^4b}\).
\end{enumerate}

Here we see another drawback of group presentations. If we didn't know the existence of \(D_5\), then it would have been harder to classify groups of order \(10\). It is generally hard to find contradicting relations, and hard to check associativity between group elements.

\rmk It is known that the presentation \(\span{a, b \mid a^m = b^n = 1, ba = a^rb}\) gives a group of order \(mn\) if and only if \(r^n \equiv 1 \pmod m\).\footnote{Check Exercise 40.13.}

\ex. \(\abs{G} = 8\).
\begin{enumerate}
    \item If \(G\) is abelian, then \(G\) is isomorphic to either \(\Z_8\), \(\Z_2 \times \Z_4\) or \(\Z_2 \times \Z_2 \times \Z_2\).
    \item If \(G\) is not abelian, elements of \(G\) have order \(1, 2, 4, 8\). If all elements (except \(1\)) have order \(2\), then \(G\) is abelian.\footnote{\(a = a\inv\) for all \(a \in G\). This was an exercise in the previous semester.} If some element has order \(8\), then \(G\) is cyclic and abelian.

    Thus there exists an element \(a \in G\) with order \(4\). Let \(H = \span{a}\). Take \(b \in G \bs H\), then similarly, \(b^2 \in H\). Thus \(b^2 \) is equal to either \(1\) or \(a^2\).\footnote{Other cases are impossible. Check the order of \(b\).} Now we know that \(G = \span{a, b} = \{1, a, a^2, a^3, b, ab, a^2b, a^3b\}\). Using similar methods, \(ba\) must be either \(a^2b\) or \(a^3b\).

          Recall that \(H \nsub G\), so \(bHb\inv = H\). Then \(bab\inv\) has order \(4\) since
          \[
              H = \span{bab\inv} = \{e, bab\inv, ba^2b\inv, ba^3b\inv, ba^4b\inv\}.
          \]
          Then \(bab\inv\) is either \(a\) or \(a^3\). Thus \(ba = ab\) or \(ba = a^3b\), but \(G\) is not abelian, so \(ba = a^3b\). Therefore, \(G\) is isomorphic to either
          \begin{center}
              \(\span{a, b \mid a^4 = b^2 = 1, ba = a^3b}\) \quad or \quad \(\span{a, b \mid a^4 = 1, b^2 = a^2, ba = a^3b}\).
          \end{center}
          We already know to non-isomorphic to order \(8\) non-abelian groups, \(D_4\) and \(Q_8\). The former is \(D_4\) and latter is \(Q_8\).
\end{enumerate}

It is generally hard to classify groups using this way. But we have no other tools, so it is sometimes useful!

\pagebreak

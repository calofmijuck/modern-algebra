\setcounter{chapter}{8}
\chapter{Factorization}

Given a field \(F\), \(F[x]\) is an integral domain. We know that \(f(x) \in F[x]\) can be factorized as \(f(x) = p_1(x)p_2(x) \cdots p_r(x)\) where \(p_i(x)\) are irreducible. Also, this representation is unique up to order and constant factors.

Given an integral domain \(D\), we can consider the \textit{field of quotients} \(Q_D\) of \(D\). We learned that for \(f(x) \in \Z[x]\),
\begin{center}
    \(f(x) = g(x)h(x) \in \Q[x] \iff f(x) = g_1(x)h_1(x) \in \Z[x]\)
\end{center}
where \(\deg g = \deg g_1\), \(\deg h = \deg h_1\). We want to see if this also works for \(D\) and \(Q_D\). As such, we will learn what kinds of properties hold on general integral domains.

\medskip

For this part, we fix \(R\) (and other rings) to be a \textbf{commutative ring with 1}.

\setcounter{topic}{44}
\topic{Unique Factorization Domains}

\defn. Let \(a, b \in R\).
\begin{enumerate}
    \item If there exists \(c \in R\) such that \(b = ac\), then \(a\) \textbf{divides} \(b\), and \(a \mid b\).
    \item \(u \in R\) is a \textbf{unit} of \(R\) if \(u\) has a multiplicative inverse. i.e, \(u \mid 1\).
    \item \(a\) and \(b\) are \textbf{associates} in \(R\) is there exists a unit \(u \in R\) such that \(a = ub\).
\end{enumerate}

\ex.
\begin{enumerate}
    \item If \(a\) and \(b\) are associates, then \(a \mid b\) and \(b \mid a\).
    \item If \(a, b\) are nonzero elements of a field, then \(a\) and \(b\) are associates.
    \item In \(\Z\), \(\pm 1\) are units and \(\pm n\) are associates.
\end{enumerate}

We must factor elements into irreducible elements.

\pagebreak

\defn. \note{Irreducible Element} Let \(D\) be an integral domain. A non-unit, nonzero element \(p \in D\) is an \textbf{irreducible} if for any factorization \(p = ab\), \(a\) or \(b\) is a unit in \(D\).

Consider \(-12 \in \Z\). We know that
\[
    -12 = (-2)\cdot 2 \cdot 3 = 2 \cdot (-2) \cdot 3 = 2 \cdot 2 \cdot (-3) = (-2) \cdot (-2) \cdot (-3).
\]
We see that factorization is not unique. But since \(-1\) is a unit, it seems that factorization is unique \textit{up to unit factors}. Also, units and \(0\) can be factorized into many ways.

\defn. \note{Unique Factorization Domain} Let \(D\) be an integral domain. \(D\) is a \textbf{unique factorization domain} (UFD) if
\begin{enumerate}
    \item Every nonzero, non-unit element in \(D\) can be factorized into a finite product of irreducible elements in \(D\).
    \item For nonzero, non-unit element \(d \in D\), if
    \begin{center}
        \(d = p_1p_2 \cdots p_r = q_1q_2 \cdots q_s\) where \(p_i, q_i\) are irreducibles,
    \end{center}
    then \(r = s\) and \(p_i\) and \(q_i\) are associates by proper rearrangements.
\end{enumerate}

\ex. \(\Z\) and \(F[x]\) are both UFDs.

We know that we can use the \textit{division algorithm} on \(\Z\) and \(F[x]\). This property comes from the fact that these two rings are PIDs. Then the properties of PIDs will give UFDs. Our goal in this section is to prove that every PID is a UFD.

\defn. \note{Principal Ideal Domain} Let \(D\) be an integral domain. \(D\) is a \textbf{principal ideal domain} (PID) if every ideal in \(D\) is principal. i.e, all ideals can be generated by a single element.

\lemma. \note{Ascending Chain Condition (ACC) for PIDs} Let \(D\) be a PID. Consider a chain of ideals \(N_1 \subset N_2 \subset N_3 \subset \cdots\).
There exists \(r \in D\) such that \(N_r = N_s\) for all \(s \geq r\).

\pf Consider \(N = \bigcup N_i \nsub D\). Then \(N = \span{a}\) for some \(a \in D\), so some \(N_r\) must contain \(a\). Then for all \(s \geq r\), \(\span{a} \subset N_r \subset N_s \subset N = \span{a}\), and we have \(N_r = N_s\). \qed

\thm. Let \(D\) be a PID. Every nonzero, non-unit element \(d \in D\) is a product of irreducibles.

\pf Suppose that \(d\) is not an irreducible. Then \(d = a_1b_1\) for non-units \(a_1, b_1 \in D\). Then we have \(\span{d} \subsetneq \span{a_1}\).\footnote{If \(\span{d} = \span{a_1}\), then \(d\) and \(a_1\) are associates and \(b_1\) is a unit.} Consider a \textit{maximal} chain \(\span{d} \subsetneq \span{a_1} \subsetneq \span{a_2} \subsetneq \cdots \subsetneq \span{a_r} \subsetneq D\). Then \(a_r\) must be an irreducible. Let \(p_1 = a_r\), then \(d = p_1 d_1\) for some \(d_1 \in D\). Similarly, find an irreducible factor \(p_2\) of \(d_1\) and continue this process. If we let \(d_i = p_{i+1}d_{i+1}\), then \(\span{d} \subsetneq \span{d_1} \subsetneq \span{d_2} \subsetneq \cdots \subsetneq \span{d_{r}}\) is finite by the ACC lemma. \(d_r\) must be an irreducible, and \(d = p_1 p_2 \cdots p_rd_r\), which is a product of irreducibles. \qed

\pagebreak

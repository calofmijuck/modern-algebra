\defn. \(f(x) \in F[x]\) \textbf{splits} in \(E\) if \(f(x)\) factors into a product of linear factors in \(E[x]\).

\cor. Let \(E\) be an extension field of \(F\). If \(E\) is a splitting field of \(F\), then any irreducible polynomial in \(F[x]\) having a zero in \(E\) splits in \(E\).

\pf From the proof of the above theorem, if an irreducible polynomial \(f(x) \in F[x]\) has a zero in \(E\), then all other zeros must also be in \(E\). Thus \(f(x)\) splits in \(E\).

Recall that \(\Q(\sqrt[3]{2})\) is not a splitting field of \(x^3-2\). Now we can omit the polynomial and just say that it is not a splitting field. \(x^3 - 2 = (x - \sqrt[3]{2})(x^2 + \sqrt[3]{2}x + \sqrt[3]{2}^2)\) cannot be factored further into linear terms on \(\Q(\sqrt[3]{2})\).

\cor. Let \(E \leq \bar{F}\) be a splitting field of \(F\). Then every isomorphism from \(E\) onto a subfield of \(\bar{F}\) leaving \(F\) fixed is actually an automorphism of \(E\). In particular, if \(E\) has finite degree, \(\{E : F\} = \abs{G(E \quotient F)}\).

\pf Extend the isomorphism to an automorphism of \(\bar{F}\), which induces an automorphism \(\tau|_E\) on \(E\) since \(E\) is a splitting field. Then by definition \(\{E : F\} = \abs{G(E \quotient F)}\).\footnote{\(\abs{G(E\quotient F)} \leq \{E : F\}\) always holds.}

\ex. If \(E\) is a splitting field of \(x^3 - 2 \in \Q[x]\), \(E = \Q(\sqrt{2}, \omega)\). We want to inspect \(G(E \quotient \Q)\), so consider the irreducible polynomials of \(\sqrt[3]{2}\) and \(\omega\), which are \(x^3 - 2\) and \(x^2 + x + 1\).

Thus \(\sqrt[3]{2}\) can be mapped to either \(\sqrt[3]{2}\), \(\sqrt[3]{2}\omega\) or \(\sqrt[3]{2}\omega^2\), and \(\omega\) can be mapped to either \(\omega\) or \(\omega^2\). Thus \(\abs{G(E \quotient \Q)} = 6\). Similarly, \(\abs{G(E \quotient \Q(\sqrt[3]{2}))} = 2\), but \(\abs{G(\Q(\sqrt[3]{2}) \quotient \Q)} = 1\) since only the identity is possible.

Recall that \(\{E : \Q\} = \{E : \Q(\sqrt[3]{2})\} \{\Q(\sqrt[3]{2}) : \Q\}\). However, \(\abs{G(\Q(\sqrt[3]{2}) \quotient \Q)} \neq \{\Q(\sqrt[3]{2}) : \Q\}\), since \(\Q(\sqrt[3]{2})\) is not a splitting field.

\rmk Why was \(\{\Q(\sqrt[3]{2}) : \Q\} = 3\)? We know that \(\irr(\sqrt[2]{3}, \Q) = x^3 - 2\), which has \(3\) zeros in \(\bar{\Q}\). If \(\tau : \Q(\sqrt[3]{2}) \ra \bar{\Q}\) fixes \(\Q\), then \(\tau(\sqrt[3]{2})\) must be one of \(\sqrt[3]{2}\), \(\sqrt[3]{2}\omega\), \(\sqrt[3]{2}\omega^2\). We see that the number \(3\) could have came from the degree of the irreducible polynomial or the \(3\) zeros of the irreducible polynomial.

\question For algebraic \(\alpha \in E\), \(\{F(\alpha) : F\} = \deg(\alpha, F) = [F(\alpha) : F] = \dim_F F(\alpha)\)?

This is not always true, but it is true for some \(F\).\footnote{Answer: \(E\) is a finite normal extension of \(F\).}

\pagebreak

\topic{Separable Extensions}

Is this possible: \(\irr(\alpha, F) = (x - \alpha)^n\) for \(n \geq 2\)? Then we would have \(\{F(\alpha) : F\} = 1 \neq [F(\alpha) : F] = n\). So if \(\irr(\alpha, F)\) has distinct zeros in \(\bar{F}\), then \(\{F(\alpha) : F\} = [F(\alpha) : F]\). We first consider the multiplicity of zeros of polynomials.

\section*{Multiplicity of Zeros of a Polynomial}

\defn. \note{Multiplicity} Let \(f(x) \in F[x]\) and \(\alpha \in \bar{F}\). \(\alpha\) is zero of \textbf{multiplicity} \(\nu\) if \(\nu\) is the greatest integer such that \((x - \alpha)^\nu \mid f(x)\) in \(\bar{F}[x]\).

\thm. For irreducible \(f(x) \in F[x]\), all zeros of \(f(x)\) in \(\bar{F}\) have the same multiplicity.

\pf Let \(\alpha, \beta \in \bar{F}\) be zeros of \(f(x)\). Consider the conjugation isomorphism \(\psi_{\alpha, \beta} : F(\alpha) \ra F(\beta)\), which can be extended to an isomorphism \(\tau : \bar{F} \ra \bar{F}\), which induces \(\tau_x : \bar{F}[x] \ra \bar{F}[x]\). (\(\tau_x(x) = x\)) Note that \(\tau_x(f(x)) = f(x)\) since \(\tau\) leaves \(F\) fixed. But \(\tau_x((x-\alpha)^\nu) = (x-\beta)^\nu\), so \((x - \alpha)^\nu \mid f(x)\) implies \((x - \beta)^\nu \mid f(x)\), showing that \(\beta\) must have greater multiplicity than \(\alpha\). The reverse inequality can be shown by a similar argument on \(\psi_{\beta, \alpha}\). \qed

\cor. For irreducible \(f(x) \in F[x]\),
\[
    f(x) = c \prod (x - \alpha_i)^\nu = c \left(\prod (x - \alpha_i)\right)^\nu
\]
where \(\alpha_i\) are distinct zeros of \(f(x)\) in \(\bar{F}\) and \(c \in F\).

The following is an example where a zero has multiplicity greater than \(1\). Such situations can occur in infinite fields of characteristic \(p \neq 0\).

\ex. Let \(E = \Z_p(y)\) for an indeterminate \(y\). Take a subfield \(F = \Z_p(t)\) by setting \(t = y^p\). Then \(E = F(y)\), and \(y\) is a zero of \(x^p - t \in F[x]\), so \(E\) is an algebraic extension of \(F\) and \(\irr(y, F) \mid (x^p -t)\).  But \(x^p - t = x^p - y^p = (x - y)^p\), and since \(y \notin F\), \(\deg(y, F) > 1\). Thus \(y\) is a zero of \(\irr(y, F)\) with multiplicity greater than \(1\). \qed

\smallskip

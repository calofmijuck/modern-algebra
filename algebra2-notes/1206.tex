\topic{Insolvability of the Quintic}

\defn. \note{Radical Extension} An extension \(K\) of \(F\) is an \textbf{extension of \(F\) by radicals} if
\begin{enumerate}
    \item \(\exists \alpha_1, \dots, \alpha_r \in K\) such that \(K = F(\alpha_1, \dots, \alpha_r)\)
    \item \(\exists n_1, \dots, n_r \in \N\) such that \(\alpha_1^{n_1} \in F\) and \(\alpha_k^{n_k} \in F(\alpha_1, \dots, \alpha_{k-1})\) for \(k = 2, \dots, r\).
\end{enumerate}

In the second condition, \(\alpha_1 = \sqrt[n_1]{a}\) for some \(a \in F\). The condition for \(\alpha_k\) says that we allow nested radicals. Therefore, we can have radicals in the formula.

\defn. \note{Solvable by Radicals} \(f(x) \in F[x]\) is \textbf{solvable by radicals} over \(F\) if the splitting field \(E\) of \(f(x)\) over \(F\) is contained in a radical extension of \(F\).

The splitting field \(E\) of \(f(x)\) over \(F\) contains zeros of \(f(x)\). We want to express the zeros using \(+, -, \times, \div, \sqrt[n]{\phantom{a}}\). We can do this if there is a radical extension \(K\) over \(F\) containing \(E\).

\ex.
\begin{enumerate}
    % \item \(f(x) = x^n - 1 \in \Q[x]\).
    \item \(f(x) = ax^2 + bx + c \in \Q[x]\) with \(a, b, c \in \Q\).
    \item \(f(x) = x^n - a \in \Q[x]\) with \(a \in \Q\). The splitting field is \(E = \Q(\zeta_n, \sqrt[n]{a})\) where \(\zeta_n\) is the primitive \(n\)-th root of unity. \(E\) is a radical extension of \(F\).
\end{enumerate}

\recall A group \(G\) is \textbf{solvable} if there exists a composition series \(\{H_i\}\) such that all quotient groups \(H_i/H_{i-1}\) are abelian. For abelian \(G\), it is solvable. For \(S_n\) with \(n \geq 5\), the series
\begin{center}
    \(\{e\} < A_n < S_n\)
\end{center}
is a composition series since \(A_n\) is simple. But \(A_n\) is not abelian, so \(S_n\) is not a solvable group.

\lemma. If \(K\) is the splitting field of \(x^n - a \in F[x]\), then \(G(K/F)\) is solvable.

\pf Let \(\zeta\) be a primitive \(n\)-th root of unity.

\note{Case 1} Suppose \(\zeta \in F\), then \(K = F(\sqrt[n]{a})\) and \(\sigma \in G(K/F)\) is determined by its value on \(\sqrt[n]{a}\). Then \(\sigma(\sqrt[n]{a})\) must be \(\sqrt[n]{a}\zeta^i\) for some \(i\). Let \(\sigma_i \in G(K/F)\) as \(\sqrt[n]{a} \mapsto \sqrt[n]{a}\zeta^i\). Then \(\sigma_i \sigma_j = \sigma_j \sigma_i = \sigma_{i+j}\), showing that \(G(K/F)\) is abelian and solvable.

\note{Case 2} If \(\zeta \notin F\), then \(K = F(\zeta, \sqrt[n]{a})\). We consider the tower \(F \leq F(\zeta) \leq F(\zeta, \sqrt[n]{a})\). \(F(\zeta)\) is a cyclotomic extension of \(F\), so it is a normal extension. Also \(K\) is a normal extension of \(F\). Then by Galois theorem, we have that \(G(K/F(\zeta)) \nsub G(K/F)\). Thus the series
\begin{center}
    \(\{e\} < G(K/F(\zeta)) < G(K/F)\)
\end{center}
is a subnormal series. Next, check the quotient groups. \(G(K/F(\zeta))\) is abelian by the proof in Case 1, and again by Galois theorem,
\begin{center}
    \(G(K/F) / G(K/F(\zeta)) \simeq G(F(\zeta)/F) \simeq \Z_n\cross\)
\end{center}
is also abelian. Any composition series of \(G(K/F)\) will be a refinement of the above subnormal series. Since the quotient groups are abelian, the quotient groups in the composition series are also abelian. \(G(K/F)\) is solvable. \qed

\thm. Let \(F\) be a field of characteristic \(0\), and let \(F \leq E \leq K \leq \bar{F}\) where \(E\) is a finite normal extension over \(F\), and \(K\) is a radical extension of \(F\). Then \(G(E/F)\) is solvable.

\textit{Proof Sketch.} We are given \(F \leq E \leq K\). \(K\) is not normal, so we cannot use Galois theorem directly. Consider another \textit{normal radical} extension \(L\) of \(F\) so that \(K \leq L\). If we find a radical extension \(K\) of \(F\) containing \(E\), we can construct a normal radical extension \(L\) of \(F\). Then \(G(L/F)\) is solvable, and \(G(E/F)\) is also solvable since \(G(E/F) \simeq G(L/F)/G(L/E)\).

\pf Let \(K = F(\alpha_1, \dots, \alpha_r)\) where \(\alpha_1^{n_1} \in F\) and \(\alpha_i^{n_i} \in F(\alpha_i, \dots, \alpha_{i-1})\) for \(2 \leq i \leq r\). Form the splitting field \(L_1\) of \(f_1(x) = x^{n_1} - \alpha_1^{n_1}\) over \(F\). Then \(L_1\) is a normal extension of \(F\) and \(G(L_1/F)\) is solvable by the above lemma. Next, we have \(\alpha_2^{n_2} \in L_1\) and form the splitting field \(L_2\) of \(f_2(x)\) over \(L_1\), where
\[
    f_2(x) = \prod_{\sigma \in G(L_1/F)} \paren{x^{n_2} - \sigma(\alpha_2)^{n_2}}.
\]
\(f_2(x)\) is invariant under any \(\sigma \in G(L_1/F)\), so \(f_2(x) \in F[x]\). \(L_2\) is a splitting field over \(F\) and is a normal radical extension of \(F\). To actually construct \(L_2\), start from \(L_1\) and consider the splitting field of \(x^{n_2} - \sigma(\alpha_2)^{n_2}\) for each step, just like in the above lemma. Suppose we extend \(E_1\) to \(E_2\), then \(G(E_2/E_1)\) is solvable, and \(G(E_2/E_1) \simeq G(L_2/E_1) / G(L_2/E_2)\). So if
\[
    F \leq L_1 \leq E_1 \leq E_2 \leq \cdots \leq L_2,
\]
then
\[
    G(L_2/L_2) \leq \cdots \leq G(L_2/E_2) \leq G(L_2/E_1) \leq G(L_2/L_1) \leq G(L_2/F)
\]
where each quotient group is solvable. This is a subnormal series of a \textit{finite} group with solvable quotients, which can be refined to a composition series with abelian quotients. Thus \(G(L_2/F)\) is solvable. Continue in this manner, obtain a field \(L = L_r\) that is a normal radical extension of \(F\) and \(G(L/F)\) is solvable. By construction, \(F \leq K \leq L\).

Now, \(G(E/F) \simeq G(L/F) / G(L/E)\) is a homomorphic image of \(G(L/F)\). Since a homomorphic image of a solvable group is also solvable, \(G(E/F)\) is solvable. \qed

\rmk Suppose that \(f(x)\) is solvable by radicals over \(F\). For the splitting field \(E\) of \(f(x)\) over \(F\), \(G(E/F)\) is solvable. i.e, if \(G(E/F)\) is \textit{not solvable}, then \(f(x)\) is \textit{not solvable by radicals}.

Finally, to show the insolvability of the quintic, we need to find a polynomial \(f(x)\) such that for the splitting field \(E\) of \(f(x)\) over \(F\), \(G(E/F)\) is not solvable.

\pagebreak

\thm. Let \(y_1\) be transcendental over \(\Q\), and let \(y_i\) be transcendental over \(\Q(y_1, \dots, y_{i-1})\) for \(i = 2, 3, 4, 5\).\footnote{These \(y_i\) are called \textit{independent transcendental elements} over \(\Q\).} Let \(F = \Q(s_1, s_2, s_3, s_4, s_5)\) and consider the polynomial
\[
    f(x) = \prod_{i=1}^5 (x - y_i) \in F.
\]
The splitting field \(E\) of \(f(x)\) over \(F\) is \(\Q(y_1, y_2, y_3, y_4, y_5)\), and \(G(E/F) \simeq S_5\). Thus \(f(x)\) is not solvable by radicals over \(F = \Q(s_1, s_2, s_3, s_4, s_5)\).

\ex. Let \(f(x) \in \Q[x]\) be irreducible with \(\deg f = 5\), having \(3\) real roots \(\alpha_1, \alpha_2, \alpha_3\) and \(2\) imaginary roots \(\beta_1, \beta_2\). Let \(E\) be the splitting field of \(f(x)\) over \(\Q\). Then there exists \(\tau \in G(E/F)\) that fixes \(\alpha_i\), but swaps \(\beta_1\) and \(\beta_2\), which has order \(2\). Next, for a zero \(\alpha\) of \(f(x)\), \([\Q(\alpha) : \Q] = 5\) since \(f(x)\) is irreducible. Thus \(5 \mid [E : F]\) and \([E : F] = G(E/F)\), so \(G(E/F)\) has an element \(\sigma\) of order \(5\) by Sylow theorem.

We know that \(G(E/F) \leq S_5\). However, \(G(E/F) \simeq S_5\) since we can generate \(S_5\) from \(\tau\) and \(\sigma\), having order \(2\) and \(5\). Thus \(f(x)\) is not solvable by radicals. \qed

\pagebreak

\rmk Let \(\alpha\) be algebraic. We can extend \(id : F \ra F\) to an isomorphism \(\tau\) from \(F(\alpha)\) onto a subfield of \(\bar{F}\), then \(\tau(\alpha)\) is another zero of \(\irr(\alpha, F)\), so \(\{F(\alpha) : F\}\) is the number of distinct zeros of \(\irr(\alpha, F)\).

\thm. Let \(E\) be a finite extension of \(F\). Then \(\{E : F\} \mid [E : F]\).

% Suppose \(E = F(\alpha)\). \([E : F] = \deg(\alpha, F) = \nu \{E : F\}\), since \(\{E : F\}\) is the nubmer of distinct zeros of \(\irr(\alpha, F)\) in \(\bar{F}\). Thus \(\{E : F\} \mid [E : F]\).

\pf Let \(E = F(\alpha_1, \dots, \alpha_n)\), where \(\alpha_i \in \bar{F}\). Then
\[
    \begin{aligned}
        [F(\alpha_1, \dots, \alpha_k) : F(\alpha_1, \dots, \alpha_{k-1})] & = \deg(\alpha_k, F(\alpha_1, \dots, \alpha_{k-1}))                          \\
                                                                          & = \nu_k \{F(\alpha_1, \dots, \alpha_k) : F(\alpha_1, \dots, \alpha_{k-1})\},
    \end{aligned}
\]
where \(\nu_k\) is the multiplicity of \(\alpha_k\) in \(\irr(\alpha_k, F(\alpha_1, \dots, \alpha_{k-1}))\). Multiply the above equations for \(k = 1, \cdots, n\), then \([E : F] = \prod_i \nu_i \{E : F\}\). \qed

\section*{Separable Extensions}

We study cases where the multiplicity is \(1\).

\defn. \note{Separable Extension}
\begin{enumerate}
    \item Let \(E\) be a finite extension of \(F\). \(E\) is a \textbf{separable extension of \(F\)} if \(\{E : F\} = [E : F]\).
    \item If \(\alpha \in \bar{F}\) is \textbf{separable over \(F\)} if \(F(\alpha)\) is a separable extension.
    \item An irreducible polynomial \(f(x) \in F[x]\) is \textbf{separable} if all zeros of \(f(x)\) are separable.
    \item Let \(E\) be an algebraic extension of \(F\). \(E\) is a \textbf{separable extension of \(F\)} if every \(\alpha \in E\) is separable over \(F\).
\end{enumerate}

\rmk These definitions are closely related.
\begin{enumerate}
    \item \(\alpha \in \bar{F}\) is separable over \(F \iff \{F(\alpha) : F\} = [F(\alpha) : F] \iff \alpha\) is a zero of \(\irr(\alpha, F)\) with multiplicity \(1 \iff\) \(\irr(\alpha, F)\) has all zeros of multiplicity \(1\).
    \item An irreducible \(f(x) \in F[x]\) is separable \(\iff f(x)\) has all zeros of multiplicity \(1\).
\end{enumerate}

\thm. Let \(E\) be a finite extension of \(F\) and \(K\) be a finite extension of \(E\). \(K\) is separable over \(F\) if and only if \(K\) is separable over \(E\) and \(E\) is separable over \(F\).

\pf Since index divides the degree, use the equations
\[
    [K : F] = [K : E][E : F], \qquad \{K : F\} = \{K : E\}\{E : F\}.
\]
Then the result follows directly from the definition. \qed

\cor. Let \(E\) be a finite extension of \(F\). \(E\) is separable over \(F\) if and only if every \(\alpha \in E\) is separable over \(F\).

\pf \note{\mimp} \(F \leq F(\alpha) \leq E\), so \(F(\alpha)\) is separable over \(F\).

\note{\mimpd} Let \(E = F(\alpha_1, \dots, \alpha_n)\) for some \(\alpha_1, \dots, \alpha_n \in E\). We show that \(F(\alpha_1, \dots, \alpha_k)\) is separable over \(F(\alpha_1, \dots, \alpha_{k-1})\) for all \(k = 2, \dots, n\). Then we are done.

\(\alpha_k\) is separable over \(F\), so it is a zero of \(\irr(\alpha_k, F)\) with multiplicity \(1\). \(\irr(\alpha_k, F(\alpha_1, \dots, \alpha_{k-1}))\) divides \(\irr(\alpha_k, F)\), so \(\alpha_k\) is a zero of \(\irr(\alpha_k, F(\alpha_1, \dots, \alpha_{k-1}))\) with multiplicity \(1\). Thus \(\alpha_k\) is separable over \(F(\alpha_1, \dots, \alpha_{k-1})\). \qed

\section*{Perfect Fields}

\lemma. Let \(f(x) = x^n + a_{n-1}x^{n-1} + \cdots + a_0 \in \bar{F}[x]\). If \((f(x))^m \in F[x]\) and \(m \cdot 1 \neq 0\) in \(F\) then \(f(x) \in F[x]\).

\pf \((f(x))^m = x^{mn} + ma_{n-1}x^{mn-1} + \cdots \in F[x]\), but \(m \neq 0\), so \(a_{n-1} \in F\). Assume that \(a_{n-1}, \dots, a_{n-r} \in F\) and compute the coefficient of \(x^{mn - (r+1)}\) in \((f(x))^m\). It must be in \(F\) and \(m \neq 0\), so \(a_{n-(r+1)} \in F\).\footnote{Has the form \(ma_{n-(r+1)}+g_{r+1}(a_{n-1}, \dots, a_{n-r})\) where \(g_{r+1}\) is a polynomial expression.} By induction, all coefficients are in \(F\). \qed

When does \(\alpha \in E\) fail to be separable over \(F\)?

\defn. \note{Perfect Field} A field is a \textbf{perfect field} if every finite extension is separable.

\thm. Every field \(F\) of characteristic zero is perfect.

\pf Let \(E\) be a finite extension of \(F\), take \(\alpha \in E\). Then \(\irr(\alpha, F) = \prod (x - a_i)^\nu \in F[x]\) where \(a_i \in \bar{F}\) are distinct. Put \(a_1 = \alpha\). By the above lemma, \(\prod (x - a_i) \in F[x]\) and this polynomial also has \(\alpha\) as a zero. Thus \(\nu = 1\). \(\alpha\) is separable over \(F\) and \(E\) is separable over \(F\). \qed

\thm. Every finite field \(F\) is perfect.

\pf Read by yourself. (TBA) \qed

\rmk \(F = \Z_p(t)\), \(E = \Z_p(y)\) with \(t = y^p\) is not separable and not perfect.

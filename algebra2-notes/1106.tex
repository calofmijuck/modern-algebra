\chapter{Automorphisms and Galois Theory}

We will prove the unsolvability of the quntic. It says that there is no \textit{general formula} for solving the quntic. But what is a \textit{general formula} for solving an equation? What does this mean mathematically? We must define this first.

We will learn \textbf{Galois theory}, and use the tools we learned from group theory and field theory. It is easy to get lost, so remember the big picture!

We assume that all algebraic extensions and all algebraic elements over \(F\) are contained in one fixed algebraic closure \(\bar{F}\) of \(F\).

\topic{Automorphisms of Fields}

\defn. \note{Conjugate} Let \(E\) be an algebraic extension field of \(F\). \(\alpha, \beta \in E\) are said to be \textbf{conjugate} if \(\irr(\alpha, F) = \irr(\beta, F)\).

\ex.
\begin{enumerate}
    \item For \(a, b \in \R\), we know that \(a+bi\) and \(a-bi\) are (complex) \textit{conjugates}. Assuming that \(b \neq 0\),
          \[
              \irr(a + bi, \R) = x^2 - 2ax + (a^2 + b^2) = \irr(a - bi, \R).
          \]

    \item \(\sqrt{2}\) and \(-\sqrt{2}\) are conjugate over \(\Q\).
    \item \(\sqrt[3]{2}\), \(\sqrt[3]{2}\omega\), \(\sqrt[3]{2}\omega^2\) are conjugate over \(\Q\). It can be checked that the irreducible polynomials of the three elements are \(x^3 - 2 \in \Q[x]\).
\end{enumerate}

\rmk Factorization of polynomials and checking its irreducibility will be very important from now on.

\pagebreak

\thm. \note{Conjugation Isomorphism} Let \(F\) be a field and let \(\alpha, \beta \in E\) be algebraic over \(F\) with \(\deg(\alpha, F) = n\). Then the following are equivalent.
\begin{enumerate}
    \item There exists an isomorphism \(\psi_{\alpha, \beta} : F(\alpha) \ra F(\beta)\) defined as
          \[
              \psi_{\alpha, \beta}(c_0 + c_1 \alpha + \cdots + c_{n-1} \alpha^{n-1}) = c_0 + c_1\beta + \cdots + c_{n-1} \beta^{n-1}, \quad (c_i \in F).
          \]

    \item \(\alpha\) and \(\beta\) are conjugate over \(F\).
\end{enumerate}

\pf \note{\mimp} Let \(f(x) = \irr(\alpha, F) \in F[x]\). Then \(\psi_{\alpha, \beta}(f(\alpha)) = 0 = f(\beta)\) by definition. Thus \(\beta\) is a zero of \(f(x)\), so \(\irr(\beta, F) \mid \irr(\alpha, F)\). Next, consider \(\psi_{\alpha, \beta}\inv = \psi_{\beta, \alpha}\). By a similar argument, we have \(\irr(\alpha, F) \mid \irr(\beta, F)\). These are monic polynomials, so \(\irr(\alpha, F) = \irr(\beta, F)\).

\note{\mimpd} Recall that \(F(\alpha) \simeq F[x] \quotient \span{\irr(\alpha, F)}\). Let \(p(x) = \irr(\alpha, F) = \irr(\beta, F)\). Then \(F(\alpha) \simeq F[x] \quotient \span{p(x)} \simeq F(\beta)\) by the first isomorphism theorem.
\[
    \begin{tikzcd}
        & F[x] \arrow{d}{\pi} \arrow[bend right]{ddl}[swap]{\varphi_\alpha} \arrow[bend left]{ddr}{\varphi_\beta} & \\
        & F[x] \quotient \span{p(x)} \arrow{dl}{\simeq}[swap]{\mu_\alpha} \arrow{dr}{\mu_\beta}[swap]{\simeq} & \\
        F(\alpha) \arrow{rr}{\psi_{\alpha, \beta}}[swap]{{\simeq}} & & F(\beta)
    \end{tikzcd}
\]
We see that \(\psi_{\alpha, \beta} = \mu_\beta \circ \mu_\alpha\inv\). \qed

Since we quotient out \(F[x]\) with the same irreducible polynomial, the simple extensions \(F(\alpha)\), \(F(\beta)\) must have the same structure. Thus, the isomorphism \textit{should} be given by the mapping \(\alpha \mapsto \beta\).

\ex. Examples of conjugation isomorphisms.
\begin{enumerate}
    \item \(\psi_{i, -i} : \R(i) \ra \R(-i)\) will be defined as \(a + bi \mapsto a - bi\).
    \item \(\psi_{\sqrt[3]{2}, \sqrt[3]{2}\omega} : \Q(\sqrt[3]{2}) \ra \Q(\sqrt[3]{2}\omega)\) is defined as \(\sqrt[3]{2} \mapsto \sqrt[3]{2}\omega\) and \(a \mapsto a\).
\end{enumerate}

\cor. Let \(\alpha \in E\) be algebraic over \(F\).
\begin{enumerate}
    \item Let \(\psi : F(\alpha) \ra E \leq \bar{F}\) be any isomorphism such that \(\psi(a) = a\) for \(a \in F\). Then \(\psi\) maps \(\alpha\) onto a conjugate \(\beta\) of \(\alpha\) over \(F\).

    \item If \(\beta\) is a conjugate of \(\alpha\) over \(F\), then there exists a unique isomorphism \(\psi_{\alpha, \beta} : F(\alpha) \ra E \leq \bar{F}\) such that \(\alpha \mapsto \beta\) and \(a \mapsto a\) for \(a \in F\).
\end{enumerate}

\pf \note{1} Let \(f(x) = \irr(\alpha, E)\), then \(f(\alpha) = 0\). So \(\psi(f(\alpha)) = f(\psi(\alpha)) = 0\) and \(\beta = \psi(\alpha)\) is a conjugate of \(\alpha\).

\note{2} The isomorphism is uniquely determined by its values on \(a \in F\) and \(\alpha\). \qed

\cor. Let \(f(x) \in \R[x]\) with \(a + bi \in \C \bs \R\). If \(f(a + bi) = 0\), then \(f(a - bi) = 0\).

\pf Since \(i\) and \(-i\) are conjugate over \(\R\), consider the isomorphism \(\psi_{i, -i} : \R(i) \ra \R(-i)\). Then \(\psi_{i, -i}\paren{f(a+bi)} = f(\psi_{i, -i}(a + bi)) = f(a-bi)\). \qed

Now we study new terms and definitions.

\defn. \note{Automorphism} An isomorphism of a field onto itself is called an \textbf{automorphism} of the field.

\defn. Let \(E, E'\) be two extension fields of \(F\). Let \(\sigma : E \ra E'\) be an isomorphism.
\begin{enumerate}
    \item \(a \in E\) is \textbf{fixed} by \(\sigma\) if \(\sigma(a) = a\).

    \item If \(\sigma(a) = a\) for all \(a \in K \subset E\), we say that \(K\) is \textbf{fixed} by \(\sigma\).\footnote{Also: \(\sigma\) leaves \(K\) fixed.}
\end{enumerate}

\ex.
\begin{enumerate}
    \item \(\psi_{\sqrt{2}, -\sqrt{2}} : \Q(\sqrt{2}) \ra \Q(-\sqrt{2})\) is a field isomorphism. We see that \(\Q\) is fixed by \(\psi_{\sqrt{2}, -\sqrt{2}}\).
    \item \(\sigma : \Q(\sqrt{2}, \sqrt{3}) \ra \Q(\sqrt{2}, \sqrt{3})\) defined as \(a + \sqrt{2}b + \sqrt{3}c + \sqrt{6}d \mapsto a + \sqrt{2}b - \sqrt{3}c - \sqrt{6}d\). Then \(\Q(\sqrt{2})\) is fixed by \(\sigma\).
\end{enumerate}

\thm. Let \(\{\sigma_i : i \in \mc{I}\}\) be a collection of automorphisms of a field \(E\). The set
\[
    E_{\{\sigma_i\}} = \{a \in E : \sigma_i(a) = a,\, \forall i \in \mc{I}\}
\]
forms a subfield of \(E\).

\pf Trivial. \qed

\defn. \note{Fixed Field}
\begin{enumerate}
    \item \(E_{\{\sigma_i\}} = \{a \in E : \sigma_i(a) = a,\, \forall i \in \mc{I}\}\) is called a \textbf{fixed field} of \(\{\sigma_i\}_{i \in \mc{I}}\) in \(E\).
    \item \(E_\sigma = \{a \in E : \sigma(a) = a\}\) is called a \textbf{fixed field} of \(\sigma\) in \(E\).
\end{enumerate}

\thm. \note{Automorphism Group} The set of automorphisms of a field \(E\) is a group under composition. We will denote this group as \(\Aut(E)\).

\thm. Let \(F \leq E\).
\[
    G(E \quotient F) = \{\sigma \in \Aut(E) \mid F \text{ is fixed by } \sigma\} \leq \Aut(E).
\]
Also, \(F \leq E_{G(E \quotient F)}\).

\pf Mechanical. \(F \leq E_{G(E \quotient F)}\) holds by definition. \qed

\defn. \(G(E \quotient F)\) is the \textbf{group of automorphisms of \(E\) leaving \(F\) fixed}, or briefly the \textbf{group of \(E\) over \(F\)}.

\smallskip

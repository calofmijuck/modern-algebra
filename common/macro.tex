\newcommand{\ds}{\displaystyle}

\newcommand{\mf}[1]{\mathfrak{#1}}
\newcommand{\mc}[1]{\mathcal{#1}}
\newcommand{\bb}[1]{\mathbb{#1}}
\renewcommand{\bf}[1]{\mathbf{#1}}

\newcommand{\inv}{^{-1}}
\newcommand{\adj}{\text{*}}
\newcommand{\cross}{^\times}
\newcommand{\bs}{\setminus}
\newcommand{\nsub}{\unlhd}
\newcommand{\pnsub}{\lhd}

\newcommand{\norm}[1]{\left\lVert #1 \right\rVert}
\newcommand{\abs}[1]{\left| #1 \right|}
\newcommand{\paren}[1]{\left( #1 \right)}
\newcommand{\seq}[1]{\left\{ #1 \right\}}
\renewcommand{\span}[1]{\left\langle #1 \right\rangle}
\renewcommand{\bar}[1]{\overline{#1 \vphantom{l}}}
\newcommand{\ind}[2]{\left(#1 : #2\right)}
\newcommand{\lcm}{{\rm lcm}}

\newcommand{\ra}{\rightarrow}
\newcommand{\imp}{\implies}
\newcommand{\mimp}{\(\implies\)}
\newcommand{\mimpd}{\(\impliedby\)}
\newcommand{\miff}{\!\!\(\iff\)}
\newcommand{\mast}{\(\ast\)}
\newcommand{\mstar}{\(\star\)}

\newcommand{\N}{\mathbb{N}}
\newcommand{\Z}{\mathbb{Z}}
\newcommand{\Q}{\mathbb{Q}}
\newcommand{\R}{\mathbb{R}}
\newcommand{\C}{\mathbb{C}}

\DeclareMathOperator{\im}{im}
\DeclareMathOperator{\ch}{char}
\DeclareMathOperator{\quotient}{/}
\newcommand{\End}{\mathrm{End}}
\newcommand{\Aut}{\mathrm{Aut}}
\newcommand{\irr}{\mathrm{irr}}
\newcommand{\GF}{\mathrm{GF}}

\let\oldexists\exists
\renewcommand{\exists}{\oldexists\,}

\let\oldtilde\tilde
\renewcommand{\tilde}[1]{\widetilde{#1}}

\newcommand{\defn}[1]{%
    \ifthenelse{\equal{#1}{.}}
    {\textbf{\sffamily Definition.}}%
    {\textbf{\sffamily Definition #1.}}%
}

\newcommand{\thm}[1]{%
    \ifthenelse{\equal{#1}{.}}
    {\textbf{\sffamily Theorem.}}%
    {\textbf{\sffamily Theorem #1.}}%
}

\newcommand{\prop}[1]{%
    \ifthenelse{\equal{#1}{.}}
    {\textbf{\sffamily Proposition.}}%
    {\textbf{\sffamily Proposition #1.}}%
}

\newcommand{\ex}[1]{%
    \ifthenelse{\equal{#1}{.}}
    {\textbf{\sffamily Example.}}%
    {\textbf{\sffamily Example #1.}}%
}

\newcommand{\prob}[1]{%
    \ifthenelse{\equal{#1}{.}}
    {\textbf{\sffamily Problem.}}%
    {\textbf{\sffamily Problem #1}}%
}

\newcommand{\lemma}[1]{%
    \ifthenelse{\equal{#1}{.}}
    {\textbf{\sffamily Lemma.}}%
    {\textbf{\sffamily Lemma #1.}}%
}

\newcommand{\cor}[1]{%
    \ifthenelse{\equal{#1}{.}}
    {\textbf{\sffamily Corollary.}}%
    {\textbf{\sffamily Corollary #1.}}%
}

\newcommand{\recall}{\textbf{\sffamily Recall.\;}}
\newcommand{\rmk}{\textbf{\sffamily Remark.\;}}
\newcommand{\pf}{\textit{\sffamily Proof.\;}}
\newcommand{\question}{\textbf{\sffamily Question.\;}}
\newcommand{\notation}{\textbf{\sffamily Notation.\;}}
\newcommand{\claim}{\textbf{Claim}}

\newcommand{\note}[1]{({\sffamily #1})}
\newcommand{\sref}[1]{{\sffamily #1}}

\newcounter{topic}
\setcounter{topic}{0}
\newcommand{\topic}[1]{
    \addtocounter{topic}{1}
    \vspace{15pt}
    \phantomsection
    \addcontentsline{toc}{section}{\thetopic.\, #1}
    \textbf{\large Section \thetopic.\, #1}
    \vspace{3pt}
    \hrule
    \vspace{10pt}
}

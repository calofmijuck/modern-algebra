\subsection*{Ideal Structure of \(F[x]\)}

Let \(F\) be a field, and let \(R\) be a commutative ring with unity. This is a similar concept to cyclic groups in group theory.

\defn. \note{Principal Ideal} For \(a \in R\), \(\span{a} = aR\) is called the \textbf{principal ideal} generated by \(a\). In addition, an ideal \(N\) is prinicpal if there exists an element \(a \in R\) such that \(N = \span{a}\).

\thm. Every ideal in \(F[x]\) is principal.

\pf It is trivial that \(\span{0} = \{0\}\), and \(\span{1} = F[x]\), so let \(N\) be a nontrivial proper ideal of \(F[x]\). Take \(f(x) \in N\) which has minimal degree in \(F[x]\). Then for any \(g(x) \in N\), we can find quotient and remainder such that \(g(x) = f(x)q(x) + r(x)\), \(\deg r < \deg f\) or \(r = 0\). Then \(r(x) = g(x) - f(x)q(x) \in N\), \(r\) has to be \(0\). Thus \(N = \span{f(x)}\). \qed

\thm. Ideal \(\span{p(x)} \neq \{0\}\) of \(F[x]\) is maximal if and only if \(p(x)\) is an irreducible polynomial.

\pf \note{\mimp} If \(p(x)\) is reducible, then there exists \(r, s \in F[x]\) with degree at least 1 such that \(p(x) = r(x)s(x)\). Then, \(p(x) \in \span{r(x)}\), and \(\span{p(x)} \subsetneq \span{r(x)}\). So \(\span{p(x)}\) is not maximal.

\note{\mimpd} Suppose that \(\span{p(x)}\) is not maximal. Since all ideals are principal, there exists \(r(x) \in F[x]\) such that \(\span{p(x)} \subsetneq \span{r(x)} \subsetneq F[x]\). Then \(p(x) \in \span{r(x)}\), so \(p(x) = r(x)s(x)\) for some \(s(x) \in F[x]\). Since \(p(x)\) is irreducible, \(\deg r = 0\) or \(\deg s = 0\). If \(\deg s = 0\), then \(\span{r(x)} = \span{p(x)}\). If \(\deg r = 0\), \(r(x)\) is a nonzero constant in \(F\), so it is a unit and \(\span{r(x)} = N = F[x]\). Contradiction. \qed

\thm. Suppose that \(p(x) \in F[x]\) is irreducible. If \(p(x) \mid r(x)s(x)\), then \(p(x) \mid r(x)\) or \(p(x) \mid s(x)\).

\pf By the above theorem, \(\span{p(x)}\) has to be maximal. But maximal ideals are prime. So if \(p(x) \mid r(x)s(x)\), \(r(x)s(x) \in \span{p(x)}\). Thus \(r(x) \in \span{p(x)}\) or \(s(x) \in \span{p(x)}\), which implies \(p(x) \mid r(x)\) or \(p(x) \mid s(x)\). \qed

\rmk If \(p(x) \in F[x]\) is irreducible, \(\span{p(x)}\) is maximal, so \(E = F[x] \quotient \span{p(x)}\) is a field. Then \(F\) is isomorphic to a subfield of \(E\). For example, \(\R[x] \quotient \span{x^2 + 1} \simeq \R \oplus \R\) as a vector space, where \(r \mapsto (r, 0)\), \(rx \mapsto (0, r)\). Also, it is isomorphic to \(\C\) as fields. This will be saved for the next semester. \(E\) is called the \textbf{extension field} of \(F\) and contains zeros of \(p(x)\).

\pagebreak

\setcounter{topic}{37}
\topic{Free Abelian Groups}

What does \textit{free} mean? Suppose that we have a single element \(1\) and we want to construct an abelian group only with this element. Then we would have \(\Z\). If we had an element \(a\), then \(na\), \(-na\) (\(n \in \N\)) would be an element and we get an abelian group \(\Z a\). How about for two elements \(a, b\)? We would have \(\Z a + \Z b\).

In short, \textit{free} means that other than the abelian group and the elements \(a, b\) no other relations are required for the group. For example, \(a + b\) is just \(a + b\), not replaced by another element.

\thm. Let \(G\) be a nontrivial abelian group, \(X \subset G\). The following are equivalent.
\begin{enumerate}
    \item \(\forall a \in G\) can be uniquely expressed by \(a = n_1x_2 + n_2x_2 + \cdots + n_rx_r\), where \(x_i \in X\) and \(n_i \in \Z \bs \{0\}\).\footnote{Uniqueness is up to order, and 0 is excluded for uniqueness.}
    \item \(\span{X} = G\), \(n_1x_1 + \cdots + n_rx_r = 0\) for distinct \(x_i \in X\) if and only if \(n_1 = \cdots = n_r = 0\).
\end{enumerate}

\pf Trivial. \qed

\rmk The condition in (2) seems somewhat similar to the condition of linearly independent vectors. Recall that in linear algebra, if \(\mf{B} = \{x_1, \dots, x_r\}\) is a basis of \(V\), then
\[
    V = x_1\R \oplus \cdots \oplus x_r \R \simeq \R \oplus \cdots \oplus \R \simeq \R^r,
\]
so \(\dim V\) is all we needed to classify vector spaces. The free abelian groups also satisfy this kind of property. They have some kind of \textit{basis}.

\defn. \note{Free Abelian Group} If an abelian group \(G\) satisfies the conditions in the above theorem, \(G\) is called a \textbf{free abelian group}, and the set \(X\) is called a \textbf{basis}.

\ex.
\begin{enumerate}
    \item \(\Z \times \Z \times \cdots \times \Z\) is a free abelian group. The basis is \(\{(1, 0, \dots, 0), \dots, (0, \dots, 0, 1)\}\). Note that the basis is not unique, as it was in vector spaces.
    \item \(\Z_n\) is not free, since \(nx = 0\) for all \(x \in \Z_n\).
\end{enumerate}

\thm. Let \(G\) be a free abelian group with \(r\) basis elements. Then \(G = \Z^r\).

\pf Let \(X = \{x_1, \dots, x_r\}\) be a basis of \(G\). Consider a group homomorphism \(\varphi : G \ra \Z^r\) as \(nx_i \mapsto (0, \dots, n, 0, \dots, 0)\). (\(n\) is in the \(i\)-th component) Then \(\varphi\) is an isomorphism. \qed

\pagebreak

\thm. Let \(G\) be a free abelian group with finite basis. Then any basis of \(G\) has the same number of elements.

\pf Suppose that \(G \simeq \Z^r \simeq \Z^s\) with \(r\neq s\). Then \(G \quotient 2G \simeq \Z^r \quotient 2\Z^r \simeq \Z^s \quotient 2\Z^s\). But \(\Z^r \quotient 2\Z^r\) has order \(2^r\), while \(\Z^s \quotient 2\Z^s\) has order \(2^s\). But \(2^r \neq 2^s\), so cannot be isomorphic. \qed

These are very similar things we did in linear algebra. For vector space \(V\), the basis is not unique, but all bases have the same number of elements, and we called it the \textit{dimension} of a vector space. But in group theory, we give the name \textit{rank}. The above theorem shows that the rank is well-defined.

\defn. \note{Rank} Let \(G\) be a free abelian group. Then the \textbf{rank} of \(G\) is the number of elements in a basis of \(G\).

\subsection*{Proof of Fundamental Theorem of Abelian Groups}

\recall Let \(G\) be a finitely generated abelian group. Then
\[ \tag{\mast}
    G \simeq \Z_{p_1^{r_1}} \times \Z_{p_2^{r_2}} \times \cdots \times \Z_{p_k^{r_k}} \times \Z \times \Z \times \cdots \times \Z
\]
where \(p_i\) are primes (not necessarily distinct), \(r_i \in \N\), and this representation is unique up to order of products.

\pf \note{Sketch} \(G\) is finitely generated, so there exists a finite subset \(X \subset G\) such that \(\span{X} = G\). Suppose that \(\span{X} = \{x_1, \dots, x_n\}\). Consider \(\Z^n\) and a group homomorphism \(\varphi: \Z^n \ra G\) defined as \((a_1, \dots, a_n) \mapsto a_1 x_1 + \cdots a_n x_n\). \(\varphi\) is well-defined since \(\Z^n\) is not a quotient space, and each element is expressed uniquely.

It is enough to show that \(\varphi\) is surjective, so that \(\Z^n \quotient \ker \varphi \simeq G\), and \(\Z^n \quotient \ker \varphi \simeq (\ast)\). Since \(\ker \varphi\) is always a subgroup, we apply the lemma below, and we can show that
\[
    \Z^n \quotient \ker\varphi \simeq \Z_{d_1} \times \Z_{d_2} \times \cdots \times \Z_{d_s} \times \overbrace{\Z \times \cdots \times \Z}^{n-s}.
\]
If we factorize \(d_i\) and reorder them, we get the desired expression in (\mast). \qed

\lemma. Let \(G\) be a nontrivial free abelian group of rank \(n\). If \(\{0\} \neq K \leq G\), then \(K\) is a free abelian group of rank \(s \leq n\), and there is a basis \(\{x_1, \dots, x_n\}\) of \(G\) such that \(\{d_1 x_1, \dots, d_s x_s\}\) is a basis of \(K\) and \(d_i \mid d_{i+1}\).

\pf Omitted. (Probably next semester?) \qed
\pagebreak

\chapter{Groups and Subgroups}

\topic{Introduction and Examples}

Extending number systems.
\begin{center}
    \(\N \longrightarrow \Z \longrightarrow \Q \longrightarrow \R \longrightarrow \C\)
\end{center}

Our goal is proving the following theorem:

\thm. A polynomial of degree \(n\) is not solvable by radicals for \(n \geq 5\).

We study abstract concepts to understand concrete examples. Examples give us motivations!

\begin{itemize}
    \item Complex numbers \(\C\). \(a + bi \in \C\), \(a, b \in \R\) and \(+, \times\) defined on them.
    \item The unit circle \(U = \{a + bi : a^2 + b^2 = 1, a, b \in \R\} = \{e^{i\theta} : 0 \leq \theta < 2\pi\}\). \(U\) is not closed under addition, but closed under multiplication.

          Note that the above two representations are intrinsically the `same' representations of the unit circle. We write
          \[
              (U, \cdot) \approx \bigl([0, 2\pi), +_{2\pi}\bigr)
          \]
          and say that these two are \textbf{isomorphic}.
    \item \textbf{Roots of Unity}: \(U_n = \{\xi \in \C : \xi^n = 1\}\). We can see that
          \[
              \xi_k = e^{\frac{2\pi k}{n}i}, \quad (k = 0, \dots, n - 1).
          \]
          When we multiply two elements for example, we do the following.
          \[
              \xi_1 \cdot \xi_2 = e^{\frac{2\pi}{n}i} \cdot e^{\frac{4\pi}{n}i} = e^{\frac{6\pi}{n}i} = \xi_3
          \]
          If we look closely, we see that we have transformed elements of \((U, \cdot)\) to \(\bigl([0, 2\pi), +_{2\pi}\bigr)\). This can be done because the two sets are \textbf{isomorphic}!
\end{itemize}

\pagebreak

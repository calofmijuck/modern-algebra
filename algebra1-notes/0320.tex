\section*{March 20th, 2023}

\defn. \note{Equivalence Relation} A relation \(\mc{R}\) on \(S\) is a \textbf{equivalence relation} if it satisfies the following.
\begin{enumerate}
    \item \note{Reflexive} \(x \mc{R} x\). (\(x \in S\))
    \item \note{Symmetric} If \(x \mc{R} y\), then \(y \mc{R} x\). (\(x, y \in S\))
    \item \note{Transitive} If \(x \mc{R} y\) and \(y \mc{R} z\), then \(x \mc{R} z\). (\(x, y, z \in S\))
\end{enumerate}

\ex.
\begin{enumerate}
    \item Relation `\(=\)' on \(\Q = \left\{\frac{y}{x} : x \in \Z\cross, y \in \Z\right\}\). Defined as \(\frac{y}{x} = \frac{y'}{x'} \iff xy' = yx'\). The second equality is equality in \(\Z\). We are defining `\(=\)' in \(\Q\) using `\(=\)' in \(\Z\).
    \item Relation `\(>\)' on \(\Z\) is not symmetric, so it is not an equivalence relation.
\end{enumerate}

\thm{0.22} Equivalence relation \(\sim\) on a set \(S\) yields a partition of \(S\).

\ex. Let \(S = \Z\), \(x \sim y \iff x \equiv y \pmod 5\). Then
\[
    \Z = \bar{0} \sqcup \bar{1} \sqcup \bar{2} \sqcup \bar{3} \sqcup \bar{4},
\]
where \(\bar{x} = \{y \in \Z : x \sim y\}\).

\topic{Subgroups}

\defn. \note{Subgroup} Let \((G, *)\) be a group, \(H \subset G\). \(H\) is a \textbf{subgroup} of \(G\) if \((H, * \mid_{H\times H})\) is also a group. We write \(H \leq G\).

\ex.
\begin{enumerate}
    \item \((\Z, +) \leq (\Q, +) \leq (\R, +)\).
    \item \note{Trivial Subgroup} \(\{e\} \leq G\).
    \item \note{Improper Subgroup} \(G \leq G\).
    \item \(\{e\} \leq \Z_2 \leq \Z_4\).
    \item \(V_4 \not\simeq \Z_4\) (different subgroup lattices).
    \item \(\bf{SL}_n(\R) \leq \bf{GL}_n(\R)\).
\end{enumerate}

The following is a method to check that \(H\) is a subgrouop of \(G\).

\thm. Let \(G\) be a group and \(H \subseteq G\). Then \(H \leq G\) if and only if
\begin{enumerate}
    \item \(H\) is closed under the binary operation \(*\) on \(G\).
    \item Identity \(e \in H\).
    \item For all \(x \in H\), there exists an inverse \(x\inv \in H\).
\end{enumerate}

How can we find non-trivial subgroups? We include an element and generate elements, since the binary operations are always closed!

\thm. Let \(G\) be a group, and let \(a \in G\). Then
\[
    \{a^n : n \in \Z\} \leq G.
\]

\pf Let \(H = \{a^n : n \in \Z\}\). Then for \(a^n, a^m \in H\) (\(n, m \in \Z\)), \(a^na^m = a^{n+m} \in H\) (closed), \(e = a^0 \in H\) (has an identity), \((a^n)\inv = a^{-n} \in H\) (has an inverse). So \(H \leq G\).

\rmk Let \(H = \{a^n : n \in \Z\}\).
\begin{enumerate}
    \item \(H\) is the smallest subgroup of \(G\) containing \(a\).
    \item Any subgroup of \(G\) containing \(a\) has \(H\) as a subgroup.
    \item \(H\) is commutative.
\end{enumerate}

\defn. \note{Cyclic} Let \(G\) be a group and let \(H = \{a^n : n \in \Z\} \leq G\) for \(a \in G\).
\begin{enumerate}
    \item \(H\) is called the \textbf{cyclic subgroup} generated by \(a\), and we write \(H = \span{a}\).
    \item If there exists \(x \in G\) such that \(G = \span{x}\), then \(G\) is called a \textbf{cylic group}.
\end{enumerate}

\ex. \(U_n = \{\xi \in \C : \xi^n = 1\} = \span{\xi}\), is a cyclic group where \(\xi = e^{\frac{2\pi i}{n}}\). If we visualize this on the complex plane, the element \(\xi\) generates the whole group, in a cycle, hence the name cyclic group.
\pagebreak

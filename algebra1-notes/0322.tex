\section*{March 22nd, 2023}

\topic{Cyclic Groups}

\thm. Every cyclic group is commutative.

Always consider \(U_n \simeq (\Z_n, +_n)\) as an example, when dealing with cyclic groups.

\thm. A subgroup of a cyclic group is also cyclic.

\pf Let \(G = \span{g}\) be a cyclic group, and let \(H \leq G\). Choose the smallest positive \(r \in \N\) such that \(g^r \in H\). We show that \(H = \span{g^r}\). It is clear that \(\span{g^r} \leq H\), since \(H\) is a subgroup.

Suppose that there exists \(s \in \Z\) such that \(g^s \in H\), but \(g^s \notin \span{g^r}\). Then there exists unique quotient and remainder \(q \in \Z, t \in \{1, \cdots, r - 1\}\) such that \(s = rq + t\). Then \(g^s, g^{rq} \in H\), so \(g^s(g^{rq})\inv = g^t \in H\), contradicting the minimality of \(r\). Thus \(H \leq \span{g^r}\).

\ex. \((\Z, +) = \span{1}\). So any subgroup of \((\Z, +)\) should be \(\span{n} = n\Z\) for some \(n \in \Z\).

\defn. Let \(S \subset G\). Then \(\span{S}\) is the smallest subgroup of \(G\) generated by \(S\).

If \(H = \span{a, b} \leq \Z\), we can rewrite \(H = \span{g}\) for some \(g \in \Z\). We know that \(g = \gcd(a, b)\).

\defn. \note{Greatest Common Divisor} Let \(r, s \in \N\). The \textbf{greatest common divisor} of \(r, s\) is the generator \(d\) which generates \(\span{r, s} < \Z\). We write \(d = \gcd(a, b)\) and if \(\gcd(r, s) = 1\), we say that \(r, s\) are \textbf{relatively prime}.

\rmk \(\gcd(r, s) = 1 \iff mr + ns = 1\) for some \(m, n \in \Z\).

We want to classify all cyclic groups!

\thm. \note{Classification of Cyclic Groups} Suppose that \(G = \span{g}\) is cyclic.
\begin{enumerate}
    \item \(\abs{G} = \infty \iff G \simeq \Z\).
    \item \(\abs{G} = n \iff G \simeq \Z_n\).
\end{enumerate}

\pf \\
\note{1} Consider the map \(\varphi: G \ra \Z\), defined as \(\varphi(g) = 1\). We first check that \(\varphi\) is well-defined. This is clear, since \(\abs{G} = \infty\), so \(n \neq m \in \Z \iff g^n \neq g^m\). Otherwise, \(\abs{G}\) would be finite. This also implies that \(\varphi\) is bijective. Also \(\varphi\) is a homomorphism, since \(\varphi(g^n g^m) = m + n = \varphi(g^n) + \varphi(g^m)\). \(\varphi\) is an isomorphism and \(G \simeq \Z\).

\note{2} Consider the map \(\varphi_n: G \ra \Z_n\), defined as \(\varphi_n(g) = 1\). We can check that \(\varphi_n\) is a well-defined isomorphism.

\thm. Let \(G = \span{g}\) be a cyclic group of order \(n\).
\begin{enumerate}
    \item Let \(b = a^s \in G\). Then \(\abs{\span{b}} = \dfrac{n}{\gcd(n, s)}\).
    \item \(\span{a^s} = \span{a^t} \iff \gcd(n, s) = \gcd(n, t)\).
\end{enumerate}

\pf
\note{1} We want to find the smallest positive integer \(m\) such that \(b^m = e\). (\(a^{ms} = e\), so \(n \mid ms\)) Take \(d = \gcd(n, s)\), then \(\gcd\paren{\frac{n}{d}, \frac{s}{d}} = 1\). If \(\frac{n}{d} \mid \frac{s}{d}\cdot m\), then \(\frac{n}{d} \mid m\). Hence the smallest positive integer \(m\) is \(\frac{n}{d}\).

\chapter{Permutations, Cosets and Direct Products}

\setcounter{topic}{7}
\topic{Groups of Permutations}

\defn. \note{Permutation} A \textbf{permutation} on a set \(A\) is a bijective function \(\varphi: A \ra A\).

\rmk Let \(S_A\) be the set of permutations on \(A\). Then \(f \circ g \in S_A\) for all \(f, g \in S\), \(\circ\) has associativity, and \(id \in S_A\), \(f\inv \in S_A\) for all \(f \in S\). Therefore \((S, \circ)\) is a group.

We study the case when \(A\) is a finite set, i.e, \(A = \{1, 2, \dots, n\}\).

\pagebreak

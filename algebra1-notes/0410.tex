\section*{April 10th, 2023}

\cor. \(\prod_{i=1}^n \Z_{m_i} = \Z_{m_1} \times \cdots \Z_{m_n} \simeq \Z_{m_1 m_2 \cdots m_n} \iff \gcd(m_i, m_j) = 1\) for all \(i \neq j\).

\ex. Let \(n = p_1^{r_1}p_2^{r_2} \cdots p_k^{r_k}\) where \(p_i\) are distinct primes, \(r_i \in \N\). Then
\[
    \Z_n \simeq \Z_{p_1^{r_1}} \times \Z_{p_2^{r_2}} \times \cdots \times \Z_{p_k^{r_k}}
\]
since \(p_i^{r_i}\) are pairwise coprime.

\defn. \note{Least Common Multiple} Let \(r_1, r_2, \dots, r_m \in \N\). The \textbf{least common multiple} \(l\) of \(r_1, r_2, \dots, r_m\) is defined as the positive \(l\) such that
\[
    \span{l} = \span{r_1} \cap \span{r_2} \cap \cdots \cap \span{r_m}
\]
in \(\Z\). We write \(l = \lcm(r_1, r_2, \dots, r_n)\).

\thm. Let \((a_1, \dots, a_n) \in \prod_{i=1}^n G_i\), where each \(a_i\) has finite order \(r_i\). Then
\[
    \abs{(a_1, \dots, a_n)} = \lcm(r_1, \dots, r_n).
\]

\pf Homework! \qed

\ex. \((8, 4, 10) \in \Z_{12} \times \Z_{60} \times \Z_{24}\). Then \(r_1 = 3\), \(r_2 = 15\), \(r_3 = 12\). We want to find the smallest positive \(N\) such that \(N (8, 4, 10) = (0, 0, 0)\). Then \(3, 15, 12 \mid N\), so \(N = \lcm(3, 5, 12) = 60\).

\pagebreak

\subsection*{Structure of Finitely Generated Abelian Groups}

A group is cyclic if it can be generated by a single element. Consider the Klein 4-group \(V_4 = \Z_2 \times \Z_2\). This group has no element of order 4, so it is not cyclic. But we see that \(V_4 = \span{(1, 0), (0, 1)}\), so \(V_4\) is generated by 2 elements. We extend this definition.

\defn. \note{Finitely Generated} A group \(G\) is \textbf{finitely generated} if there exists a finite subset \(S \subset G\) such that \(G = \span{S}\).

\rmk If \(G\) is finite, \(G = \span{G}\) so it is finitely generated. But the converse is not true, since \(\Z\) is infinite but \(\Z = \span{1}\).

We started learning from the simplest groups. They were generated by a single element and we classified them. We also learned about permutation groups and Cayley's theorem. Next, we classify a large family of groups. They are finitely generated abelian groups.

\thm. \note{Fundamental Theorem of Finitely Generated Abelian Groups} Let \(G\) be a finitely generated abelian group. Then
\[
    G \simeq\Z_{p_1^{r_1}} \times \Z_{p_2^{r_2}} \times \cdots \times \Z_{p_k^{r_k}} \times \overbrace{\Z \times \Z \times \cdots \times \Z}^{\text{finite}}
\]
where \(p_i\) are primes (not necessarily distinct), \(r_i \in \N\), and this representation is unique up to order of products.

We only know cyclic groups, since they were the only groups that we classified completely. We studied permutation groups, but we saw that they are complex! So we don't know them very well, which leaves us to try a lot of things with cyclic groups. So we construct new groups from cyclic groups using direct products. Then we get this result that finitely generated abelian groups are actually a product of cyclic groups!

\rmk \(\Z_4 \not\simeq \Z_2 \times \Z_2\). If these two groups were isomorphic, then it contradicts the uniqueness of product representation. Similarly, \(\Z_9 \not\simeq \Z_3 \times \Z_3\) and \(\Z_9 \times \Z_3 \not\simeq \Z_3 \times \Z_3 \times \Z_3\).

This is a typical exercise after learning this theorem.

\ex. Classify all abelian groups of order 360.

\pf This group is finitely generated. We know that \(360 = 2^3 \times 3^2 \times 5\).
\begin{center}
    \begin{tabular}{cc}
        \(\Z_{2^3} \times \Z_{3^2} \times \Z_5\)                           & \(\Z_{2^3} \times \Z_3 \times \Z_3 \times \Z_5\)                           \\
        \(\Z_{2} \times \Z_{2^2} \times \Z_{3^2} \times \Z_5\)             & \(\Z_{2} \times \Z_{2^2} \times \Z_3 \times \Z_3 \times \Z_5\)             \\
        \(\Z_{2} \times \Z_{2} \times \Z_{2} \times \Z_{3^2} \times \Z_5\) & \(\Z_{2} \times \Z_{2} \times \Z_{2} \times \Z_3 \times \Z_3 \times \Z_5\)
    \end{tabular}
\end{center}
So these are all possible cases. \qed

\defn. \note{Decomposable} A group \(G\) is \textbf{decomposable} if \(G \simeq H \times K\) for \(H, K \leq G\).

This implies two things: that we can understand \(G\) by studying \(H, K\), and that it is important to study indecomposable groups.

\thm. A finite indecomposable abelian group is a cyclic group of order \(p^r\) where \(p\) is prime and \(r \in \N\).

\thm. Let \(G\) be a finite abelian group. If \(m \mid \abs{G}\), then there exists a subgroup of \(G\) with order \(m\).

\pf Let \(G \simeq \Z_{p_1^{r_1}} \times \Z_{p_2^{r_2}} \times \cdots \times \Z_{p_k^{r_k}}\). We use the fact that \(\Z_{p_i^{s_i}} \leq \Z_{p_i^{r_i}}\) if \(s_i \leq r_i\). Since \(m \mid \abs{G}\), we can write \(m = p_1^{s_1}p_2^{s_2} \cdots p_k^{s_k}\) for some \(s_i \leq r_i\). Take \(H_i \leq \Z_{p_i^{r_i}}\) such that \(\abs{H_i} = p_i^{s_i}\). Then \(H_1 \times H_2 \times \cdots H_k \leq G\), and it has order \(m\). \qed

Recall that from Lagrange's theorem, for finite group \(G\), if \(H \leq G\), \(\abs{H} \mid \abs{G}\). We could ask if the converse is true. \textit{Is there a subgroup of order \(m\) that divides \(\abs{G}\)?} This is not true in general, but if \(G\) is abelian this is true.

\thm. Let \(m\) be a square-free integer.\footnote{\(\nexists\, p\) prime such that \(p^2 \mid m\).} Then an abelian group of order \(m\) is cyclic.

\pf Let \(m = p_1 \cdots p_k\) where \(p_i\) are distinct primes. Using the fundamental theorem, an abelian group of order \(m\) can be written as \(\Z_{p_1} \times \Z_{p_2} \times \cdots \times \Z_{p_k}\), which is isomorphic to \(\Z_m\). \qed

\pagebreak

\setcounter{topic}{12}
\chapter{Homomorphisms and Factor Groups}

\topic{Homomorphisms}

We already learned about homomorphisms.

\recall Let \(G\), \(G'\) be groups. A map \(\varphi: G \ra G'\) such that
\begin{center}
    \(\varphi(g_1 g_2) = \varphi(g_1) \varphi(g_2)\), \quad \(\forall g_1, g_2 \in G\)
\end{center}
is called a \textbf{group homomorphism}. If \(\varphi\) is a bijection, \(\varphi\) is a \textbf{group isomorphism}.

\defn. Let \(\varphi: G \ra G'\) be a group homomorphism.
\begin{enumerate}
    \item \note{Kernel} The \textbf{kernel} of \(\varphi\) is
    \[
        \ker \varphi = \{x \in G : \varphi(x) = e'\} = \varphi\inv(\{e'\}),
    \]
    where \(e'\) is the identity of \(G'\).
    \item \note{Image} The \textbf{image} of \(\varphi\) is
    \[
        \im \varphi = \varphi(G) = \{\varphi(x) : x \in G\}.
    \]
\end{enumerate}

\pagebreak

\chapter{Permutations, Cosets and Direct Products}

\section*{March 27th, 2023}

\setcounter{topic}{7}
\topic{Groups of Permutations}

\defn. \note{Permutation} A \textbf{permutation} on a set \(A\) is a bijective function \(\varphi: A \ra A\).

\rmk Let \(S_A\) be the set of permutations on \(A\). Then \(f \circ g \in S_A\) for all \(f, g \in S\), \(\circ\) has associativity, and \(id \in S_A\), \(f\inv \in S_A\) for all \(f \in S\). Therefore \((S, \circ)\) is a group.

We study the case when \(A\) is a finite set, i.e, \(A = \{1, 2, \dots, n\}\).

\defn. \note{Symmetric Group \(S_n\)} Let \(A = \{1, 2, \dots, n\}\). Let \(S_n\) be the set of all permutations on \(A\). Then \((S_n, \circ)\) is called the \textbf{symmetric group on \(n\) letters}.

Let \(B\) be a set with \(n\) elements. We denote \(S_B\) be the set of all permutations on \(B\). With the composition operation \(\circ\), we see that \(S_B \simeq S_n\) as groups.

\ex.
\begin{enumerate}
    \item \(S_2 = \{e, \tau\}\) where \(\tau = (1, 2)\).
    \item On \(S_3\), there are 6 permutations.
          \[
              S_3 = \{e, \mu_1, \mu_2, \mu_3, \rho_1, \rho_2\}
          \]
          where \(\mu_i\) swaps other two elements other than \(i\), and \(\rho_1 = (1, 2, 3), \rho_2 = (1, 3, 2)\).
    \item Also we can see that \(S_3\) is the group of symmetries on an equilateral triangle. Each \(\rho_i\) represents a rotation, and each \(\mu_i\) represents a reflection.
\end{enumerate}

\rmk \(S_3\) is not commutative. Check that \(\rho_1 \circ \mu_1 = \mu_3\), but \(\mu_1 \circ \rho_1 = \mu_2\). In fact, \(S_3\) is the non-commutative group having the smallest order.

A natural question arises here: \textit{Can we get \(S_n\) from the symmetries of a regular \(n\)-gon?}

\ex. We try this for \(S_4\), but this doesn't work. Symmetries of a square consists of \(4\) rotations and \(4\) reflections, which is a total of 8 elements, but \(\abs{S_4} = 4! = 24\).

\defn. \note{Dihedral Group \(D_n\)} The group of symmetries of a regular \(n\)-gon is called the \textbf{\(n\)-th dihedral group} \(D_n\).

\rmk
\begin{enumerate}
    \item \(D_3 \simeq S_3\), \(D_4 < S_4\), \(D_4\) is not commutative, so \(S_4\) is not commutative.
    \item \(\abs{D_n} = 2n\).
    \item \(D_4\) is generated by 2 elements. \(D_4 = \span{\rho, \mu}\), where \(\rho\) is a rotation by 90 degrees, and \(\mu\) is some reflection.
    \item Subgroup lattice of \(D_4\).
          \begin{center}
              \begin{tikzpicture}
                  \node (D4) at (0, 4) {\(D_4\)};

                  \node (r2-f) at (-3.5, 2) {\(\span{(1, 3)(2, 4), (1, 2)(3, 4)}\)};
                  \node (r) at (0, 2) {\(\span{(1, 2, 3, 4)}\)};
                  \node (r2-rf) at (3.5, 2) {\(\span{(1, 3)(2, 4), (1, 3)}\)};

                  \node (f) at (-6, 0) {\(\span{(1, 2)(3, 4)}\)};
                  \node (r2f) at (-3.5, 0) {\(\span{(1, 4)(2, 3)}\)};
                  \node (r2) at (0, 0) {\(\span{(1, 3)(2, 4)}\)};
                  \node (rf) at (3.5, 0) {\(\span{(1, 3)}\)};
                  \node (r3f) at (6, 0) {\(\span{(2, 4)}\)};

                  \node (e) at (0, -2) {\(\{e\}\)};

                  \draw (D4) -- (r2-f.north);
                  \draw (D4) -- (r);
                  \draw (D4) -- (r2-rf.north);

                  \draw (r2-f.south) -- (f.north);
                  \draw (r2-f.south) -- (r2f.north);
                  \draw (r2-f.south) -- (r2.north);

                  \draw (r) -- (r2);

                  \draw (r2-rf.south) -- (r2.north);
                  \draw (r2-rf.south) -- (rf.north);
                  \draw (r2-rf.south) -- (r3f.north);

                  \draw (f.south)  -- (e);
                  \draw (r2f.south) -- (e);
                  \draw (r2.south) -- (e);
                  \draw (rf.south) -- (e);
                  \draw (r3f.south)  -- (e);
              \end{tikzpicture}
          \end{center}
\end{enumerate}

\lemma. For a group homomorphism \(\varphi: G \ra H\), \(\im \varphi \leq H\). Additionally if \(\varphi\) is injective, \(G \simeq \im \varphi \leq H\).

\pf \(\im\varphi\) is closed under the binary operation on \(H\), since for any \(a, b \in G\), \(\varphi(a)\varphi(b) = \varphi(ab)\), and \(ab \in G\), so \(\varphi(a)\varphi(b) \in H\). \(\varphi(e)\) is the identity, and \(\varphi(a)\inv = \varphi(a\inv)\). So \(\im\varphi \leq H\). If \(\varphi\) is injective, restricting the range of \(\varphi\) to \(\im \varphi\) gives an isomorphism, so \(G \simeq \im \varphi\).

Why do we study symmetric groups? It is because of the following theorem. It states that all groups are isomorphic to some permutation group.

\thm. \note{Cayley} Every group is isomorphic to some subgroup of \(S_n\).

\pf Consider \(\varphi : G \ra S_G\) such that \(\varphi(g) = \lambda_g\), where \(\lambda_g(x) = gx\) for \(x \in G\). (left multiplication by \(g\)) We check that \(\lambda_g \in S_G\), since groups have the cancellation law. Now check that \(\varphi\) is a monomorphism, then \(G \simeq \im \varphi \leq S_G\).

\pagebreak

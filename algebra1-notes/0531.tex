\topic{Prime and Maximal Ideals}

We assume that \(R\) is a ring with unity.

\defn. Let \(R\) be a ring and \(N \nsub R\).
\begin{enumerate}
    \item \(N \subsetneq R\) is a \textbf{proper ideal}.
    \item \(N \neq \{0\}\) is a \textbf{nontrivial ideal}.
\end{enumerate}

\ex. \(\Z / n\Z \simeq \Z_n\). Most of the examples from abelian groups apply.

\thm. Let \(R\) be a ring and let \(N\) be an ideal with a unit. Then \(N = R\).

\pf Let \(u \in N\) be a unit. Then \(u u\inv \in N\), so \(1\cdot r \in N\) for all \(r \in R\). \(N = R\). \qed

\cor. A field contains no proper nontrivial ideals.

\pf If an ideal of a field contains a nonzero element, it is a unit, so that ideal cannot be proper. \qed

\defn. \note{Maximal Ideal} \(M\) is a \textbf{maximal ideal} of a ring \(R\) if \(M\) is a proper ideal and there are no proper ideals \(N\) such that \(M \subsetneq N\).

\thm. Let \(R\) be a commutative ring with unity. Then \(M\) is a maximal ideal if and only if \(R\quotient M\) is a field.

\pf \note{\mimp} For any \(r \notin M\), any ideal containing \(r\) and \(M\) should equal \(R\). Then \(\bar{r} \in R/M\), the ideal containing \(\bar{r}\) in \(R/M\) is \(R/M\). So there exists \(\bar{r'} \in R/M\) such that \(\bar{rr'} = 1\).

\note{\mimpd} Suppose that \(M\) is not maximal. Then there exists a proper ideal \(N\) such that \(M \subsetneq N\). Then \(\varphi(N) \subset R/M\) is a nontrivial proper ideal of \(R/M\). Thus \(R/M\) is not a field. \qed

The main idea is that for the natural projection \(\varphi: R \ra R/M\), if \(I \nsub R\), then \(\varphi(I) \nsub R/M\). We have used the following lemma in the above proof.

\lemma. Let \(\varphi : R \ra R'\) be a ring homomorphism.
\begin{enumerate}
    \item If \(N \nsub R\), then \(\varphi(N) \nsub \varphi(R)\).
    \item If \(N' \nsub \varphi(R)\) or \(N' \nsub R'\), then \(\varphi\inv(N') \nsub R\).
\end{enumerate}

\pf \note{2} Take \(x \in \varphi\inv(N')\) and set \(\varphi(x) = n' \in N'\). For \(a \in R\), \(\varphi(ax) = \varphi(a)\varphi(x) \in N'\), since \(N'\) is an ideal. So \(ax \in \varphi\inv(N')\). \qed

\ex. \(\Z/n\Z \simeq \Z_n\), so \(\Z_n\) is a field if and only if \(n\Z\) is a maximal ideal of \(\Z\). Since \(\Z_n\) is a field if and only if \(n\) is prime, we conclude that \(p\Z\) are the maximal ideals of \(\Z\).

Recall the definitions of prime numbers in \(\Z\). \(p\) is prime if and only if \(p \mid ab\) for \(a, b \in \Z\), then \(p \mid a\) or \(p \mid b\). We translate this into the language of ideals.

\defn. \note{Prime Ideal} Let \(R\) be a commutative ring with unity. A proper ideal \(N\) is a \textbf{prime ideal} if \(ab \in N\) then \(a \in N\) or \(b \in N\). (\(a, b \in R\))

\ex. \(\{0\}\) is a prime ideal in an integral domain.

\thm. Let \(R\) be a commutative ring with unity. \(N\) is a prime ideal if and only if \(R/N\) is an integral domain.

\pf \(ab \in N \implies a \in N\) or \(b \in N\) if and only if \(\bar{a}\bar{b} = \bar{0} \implies \bar{a} = \bar{0}\) or \(\bar{b} = \bar{0}\). \qed

\cor. Let \(R\) be a commutative ring with unity. If \(N \nsub R\) is maximal, it is also a prime ideal.

\thm. Let \(F\) be a field. If \(\ch F = p\), \(\Z_p \hookrightarrow F\). If \(\ch F = 0\), \(\Z \hookrightarrow F\). Also, \(\Q\) is the smallest field containing \(\Z\), so \(F\) contains a subfield isomorphic to \(\Q\).

\begin{center}
    \begin{tikzpicture}[scale=0.5]
        \node (Z) at (0, 0) {\(\Z\)};
        \node (F) at (5, 0) {\(F\)};
        \node (Q) at (0, -5) {\(\Q\)};
        \draw[arrows={Hooks[right]->}, thick] (Z) -- (F) node[midway,above] {\(\varphi\)};
        \draw[arrows={Hooks[right]->}, thick] (Q) -- (F) node[midway,below left] {};
        \draw[->, thick] (Z) -- (Q) node[midway,right] {};
    \end{tikzpicture}
\end{center}

\pf Consider \(\varphi: \Z \ra F\) where \(m \mapsto m \cdot 1\), then use the 1st isomorphism theorem. \(\Q\) is the field of quotients of \(\Z\), so \(F\) must contain a subfield isomorphic to \(\Q\). \qed

Thus every field contains either a subfield isomoprhic to \(\Z_p\) for some prime \(p\), or a subfield isomorphic to \(\Q\). These fields \(\Z_p\) and \(\Q\) are building blocks.

\defn. \(\Z_p\) and \(\Q\) are called \textbf{prime fields}.

\smallskip

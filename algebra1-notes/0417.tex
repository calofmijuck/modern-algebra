\section*{April 17th, 2023}

This theorem implies that a normal subgroup is a kernel of some homomorphism.\footnote{Natural projection \(\pi : G \ra G/N\), \(\ker\pi = N\).}

\thm. \note{Fundamental Homomorphism Theorem} Let \(\varphi : G \ra G'\) be a group homomorphism. Then \(\im \varphi\) is a group, and \(\mu: G/\ker\varphi \ra \im \varphi\) defined as
\[
    \mu(a \ker\varphi) = \varphi(a), \quad a \in G,
\]
is an isomorphism.
\[
    \begin{tikzcd}
        G \arrow{rr}{\varphi} \arrow[swap]{dr}{\gamma} & \arrow[d, phantom, "\circlearrowright"]& \im \varphi \\
        & G/ \ker\varphi \arrow[dashed,swap]{ur}{\mu} &
    \end{tikzcd}
\]

\thm. If \(H \leq G\), The following are equivalent.
\begin{enumerate}
    \item \(H \nsub G\).
    \item \(gHg\inv = H\) for all \(g \in G\).
    \item \(ghg\inv \in H\) for all \(g \in G\), \(h \in H\).
    \item \(gH = Hg\) for all \(g \in H\).
\end{enumerate}

\defn. \note{Automorphism}
\begin{enumerate}
    \item An isomorphism \(\varphi : G \ra G\) is called an \textbf{automorphism}.
    \item \(i_g : G \ra G\) defined as \(i_g(x) = gxg\inv\) is called the \textbf{inner automorphism} by \(g\).
    \item \(gxg\inv\) is called a \textbf{conjugation} of \(x\) by \(g\).
    \item For \(H \leq G\), \(i_g(H) = gHg\inv\) is called the \textbf{conjugation subgroup} of \(H\).
    \item If \(i_g(H) = H\) then \(H\) is called \textbf{invariant}.
\end{enumerate}

\rmk Normal subgroups of \(G\) are invariant under all inner automorphisms.

\pagebreak

\topic{Factor Group Computations \& Simple Groups}

We want to see if some quotient group is a group we already know!

\ex.
\begin{enumerate}
    \item \(\Z_n \simeq \Z / n\Z\).
    \item \(A_n \nsub S_n\), since \(\tau A_n = A_n \tau\) for any transposition \(\tau\). (a permutation is either odd or even) Then \(\ind{S_n}{A_n} = 2\), and since a group of order 2 is unique, we have \(S_n / A_n \simeq \Z_2\).
\end{enumerate}

\thm. If \(H \leq G\) with \(\ind{G}{H} = 2\), \(H \nsub G\) and \(G/H \simeq \Z_2\).

\pf Write \(G = H \sqcup aH\) where \(a \in G \bs H\). Similarly, \(G = H \sqcup Ha\). So \(aH = Ha = G \bs H\). Therefore \(H \nsub G\), and \(G/H\) is a quotient group with order 2. Since \(\Z_2\) is the only group of order 2, \(G/H \simeq \Z_2\). \qed

\recall \note{Lagrange} Let \(G\) be a finite group with \(H \leq G\). Then \(\abs{H} \mid \abs{G}\).

\prop. The converse of Lagrange's Theorem does not hold.

\pf Consider \(A_4\). Suppose that there is a subgroup \(H\) of order 6. Then \(A_4 / H \simeq \Z_2\). Then for any \(\sigma \in A_4\), \(\sigma^2 \in H\). Since \((i, j, k)^2 = (i, k, j)\) and \((i, k, j)^2 = (i, j, k)\), every 3-cycle should be in \(H\). There are 8 3-cycles in \(A_4\), so \(\abs{H} \neq 6\). \qed

\thm. For groups \(H, K\), let \(G = H \times K\). Let \(\bar{H} = H \times \{e_K\} \leq G\). Then \(\bar{H} \nsub G\) and \(K \simeq G/\bar{H}\).

\pf \\
\note{Method 1} For any \((h, k) \in G\), \((h, k)\inv(h', e_k)(h, k) \in \bar{H}\) for all \(h, h' \in K\) and \(k \in K\). Then the map \(\varphi: K \ra G/\bar{H}\) defined as \(\varphi(k) = (\bar{e_H}, k)\) is an isomorphism. \qed

\note{Method 2} Consider \(\varphi : G \ra K\) defined as \(\varphi(h, k) = k\), and show that \(\varphi\) is an epimorphism with \(\ker \varphi = \bar{H}\). The result directly follows from the first isomorphism theorem. \qed

\ex. \((\Z_4 \times \Z_6) \quotient \span{(0, 2)}\). We see that \(\span{(0, 2)} = \{(0, 0), (0, 2), (0, 4)\}\), so the order of the quotient group should be \(24/3 = 8\). By the fundamental theorem of FGAG, this group should be isomorphic to one of \(\Z_8\) or \(\Z_4 \times \Z_2\) or \(\Z_2 \times \Z_2 \times \Z_2\).
\begin{itemize}
    \item \(\Z_4 \times \Z_6\) does not have an element of order greater than 8, so \(G/H\) doesn't either. \(G/H \not\simeq \Z_8\).
    \item For \((1, 0) \in \Z_4 \times \Z_6\), \(2(1, 0) = (2, 0) \notin \span{(0, 2)}\). So \((1, 0)\) has order greader than 2 in \(G/H\). So \(G/H \not\simeq \Z_2 \times \Z_2 \times \Z_2\).
\end{itemize}
Therefore \(G/H \simeq \Z_4 \times \Z_2\).

There are lots of exercises like this. We see another example with infinite order. We cannot enumerate all cases like above in this example.

\ex. \((\Z \times \Z) \quotient \span{(1, 1)}\). Consider \(\varphi : \Z \times \Z \ra \Z\) defined as \(\varphi(a, b) = a - b\). Next, show that \(\varphi\) is an epimorphism and that \(\ker\varphi = \span{(1, 1)}\). Then \((\Z \times \Z) \quotient \span{(1, 1)} \simeq \Z\) by the first isomorphism theorem.

Why do we study quotient groups? First, we get new kinds of groups. Second, for finitely generated abelian groups, we like to write them in their decomposed form since we know cyclic groups \textit{very} well. So we can handle them. But for non-commutative groups, we cannot use this approach. So we want `simple' groups, that will be a building block for larger groups.

\defn. \note{Simple Group} A group is \textbf{simple} if it does not have a nontrivial proper normal subgroup. i.e. only normal subgroups are \(\{e\}\) and itself.

We state this theorem and use it without proof. The insolvability of the quintic comes from this statement.

\thm. \(A_n\) is simple for \(n \geq 5\).

If simple groups are the building blocks, there should be a way to derive simple groups from any group.

\defn. \note{Maximal Normal Subgroup} \(M\) is a \textbf{maximal normal subgroup} of \(G\) if for any \textit{proper} \(N \pnsub G\), \(N \subseteq M\).

\thm. \(M\) is a maximal normal subgroup of \(G\) if and only if \(G/M\) is simple.

\pagebreak

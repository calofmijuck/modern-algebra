\setcounter{chapter}{3}
\chapter{Rings and Fields}

\section*{May 3rd, 2023}

\setcounter{topic}{17}
\topic{Rings and Fields}

\defn. \note{Ring} \((R, +, \cdot)\) is a \textbf{ring} if it satisfies the following properties.
\begin{enumerate}
    \item \((R, +)\) is an abelian group.\footnote{We will write 0 as the additive identity, \(-a\) as the additive inverse of \(a\).}
    \item \(\cdot\) is associative.
    \item \note{Distributive} \(a (b + c) = ab + ac\), \((a + b) c = ac + bc\).\footnote{The two binary operations are \textit{compatible}.}
\end{enumerate}

\rmk In our textbook, ring \(R\) may not have a multiplicative identity. Some authors require that a ring should have a multiplicative identity. We also denote the multiplicative identity as \(1\). The inverse of \(a\) is \(a\inv\) as usual.

\ex. Examples of rings.
\begin{enumerate}
    \item \((\Z, +, \cdot)\), \((\Q, +, \cdot)\), \((\R, +, \cdot)\).
    \item \(\mc{M}_{n\times n}(R)\), the set of matrices whose entries are in the ring \(R\).
    \item \(n\Z = \span{n}\) is a ring without 1, for \(n \neq 0, \pm 1\). Note that \(\Z\) and \(n\Z\) are isomorphic as groups, but it is not isomorphic as rings.
    \item \(\Z_n = \Z / n\Z\).
    \item \note{Direct Product of Rings} If \(R_1, \dots, R_n\) are rings, then \(R_1 \times \cdots \times R_n\) is also a ring, where the multiplication is done componentwise.
\end{enumerate}

\thm. Let \(a, b \in R\).
\begin{enumerate}
    \item \(0 \cdot a = a \cdot 0 = 0\).
    \item \(a(-b) = (-a)b = -ab\).
    \item \((-a)(-b) = ab\).
\end{enumerate}

\pf Exercise. \qed

Since homomorphisms and isomorphisms were defined on any binary structures, we can specialize the definition to ring homomorphisms and isomorphisms.

\defn. For rings \(R, R'\), \(\varphi: R \ra R'\) is a \textbf{ring homomorphism} if
\[
    \varphi(a + b) = \varphi(a) + \varphi(b), \quad \varphi(ab) = \varphi(a)\varphi(b), \quad \forall a, b \in R.
\]
If \(\varphi\) is bijective, then \(\varphi\) is a \textbf{ring isomorphism}.

\ex.
\begin{enumerate}
    \item \(\varphi: \Z \ra \Z_n\) defined as \(\varphi(n) = \bar{n}\) is a ring homomorphism.
    \item \(\varphi: \Z \ra 2\Z\) defined as \(\varphi(n) = 2n\) is not a ring homomorphism. \(\varphi(1) \neq \varphi(1) \varphi(1) = 4\).
\end{enumerate}

\rmk Note that if the binary operation changes, \(\varphi\) may not be a homomorphism anymore. Homomorphisms were defined on the set \textit{with the binary operation}, not on the set itself.

\prop. For \(r, s \in \Z\), if \(\gcd(r, s) = 1\), \(\Z_{rs} \simeq \Z_r \times \Z_s\) as rings.

\pf We know already that these are isomorphic as groups. Check also for multiplication, with the isomorphism \(\varphi: \Z_{rs} \ra \Z_r \times \Z_s\) defined as \(\varphi(n \cdot 1) = n \cdot (1, 1)\). \qed

\pagebreak

We cannot say much about multiplication on rings, since it only satisfies associativity. We want to consider some examples where the multiplication has more properties.

\defn. Let \(R\) be a ring.
\begin{enumerate}
    \item \note{Commutative Ring} \(R\) is a \textbf{commutative ring} if \(ab = ba\) for all \(a, b \in R\).
    \item \note{Ring with Unity} \(R\) is a \textbf{ring with unity} if \(R\) has a multiplicative identity \(1\in R\).
\end{enumerate}

We kind of hope that \((R, \cdot)\) becomes a group. We have associativity and assume a lot that it has an identity. How about inverses? But since \(0 \cdot a = 0\) for any \(a \in R\), not all elements can have inverses.

\defn. Let \(R\) be a ring with unity.
\begin{enumerate}
    \item \note{Unit} If \(u \in R\) has a multiplicative inverse \(u \inv \in R\), \(u\) is called a \textbf{unit}.
    \item \note{Division Ring} \(R\) is a \textbf{division ring} if all non-zero elements have an inverse.
    \item \note{Field} A \textbf{field} is a commutative division ring.
\end{enumerate}

\rmk For a field \(F\), \((F \bs \{0\}, \cdot)\) is a group.

\ex. Examples of fields.
\begin{enumerate}
    \item \(\Q, \R, \C\) are fields.
    \item \(\Z_p\) is a field if \(p\) is prime.
\end{enumerate}

\defn. Let \(R\) be a ring and \(F\) be a field.
\begin{enumerate}
    \item \note{Subring} If \(S \subset R\) and \(S\) is a ring by the binary operations inherited from \(R\), then \(S\) is called a \textbf{subring} of \(R\) and write \(S \leq R\).
    \item \note{Subfield} If \(K \subset F\) and \(K\) is a field by the binary operations inherited from \(F\), then \(F\) is called a \textbf{subfield} of \(F\) and write \(K \leq F\).
\end{enumerate}

\pagebreak

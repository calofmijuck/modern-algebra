\section*{March 15th, 2023}

\topic{Groups}

\defn. \note{Inverse} Let \((S, *)\) be a binary structure with identity \(e \in S\). If \(x \in S\) satisfies
\begin{center}
    \(a * x = x * a = e\) for some \(a \in S\),
\end{center}
then \(x\) is an \textbf{inverse} of \(a\), and we write \(x = a\inv\).

\defn. \note{Group} Let \((G, *)\) be a binary structure, with the following properties.
\begin{enumerate}
    \item \(*\) is associative.
    \item \(G\) has an identity element.
    \item For all \(x \in G\), there exists an inverse of \(x\) in \(G\).
\end{enumerate}
Then \(G = (G, *)\) is called a \textbf{group}.

\((\N, +)\) is not a group. The equation \(n + x = m\), \((n, m \in \N)\) does not have a solution if \(n \leq m\). So we extend the number system to \(\Z\) and consider \(n + x = m\) for \(n, m \in \Z\). This equation always has a solution, this is due to the fact that \((\Z, +)\) is a group. The operation \(+\) is associative, \(\Z\) has an identity \(0\), and also has an inverse for any \(n \in \Z\).

So if \((G, *)\) is a group, equations of the from \(a * x = b\) for given \(a, b \in G\) can be solved by multiplying \(a\inv\) on the left.
\[
    \begin{aligned}
        a\inv * (a * x) & = a\inv * b    \\
        (a\inv * a) * b & = a\inv * b  \\
        e * b           & = a\inv * b  \\
        x               & = a\inv * b.
    \end{aligned}
\]
Note that all three properties of the group were used!

\ex.
\begin{enumerate}
    \item \((\Z, +), (\Q, +), (\R, +), (\C, +)\) are all groups.
    \item \((\N, +), (\Z, \cdot), (\Q, \cdot), (\R, \cdot), (\C, \cdot)\) are not groups, since they don't have an inverse for \(0\).
    \item \((\Q\cross, \cdot), (\R\cross, \cdot), (\C\cross, \cdot)\) are groups.
    \item The roots of unity with multiplication form a group.
\end{enumerate}

\defn. \note{Commutative Group} A group \(G\) is \textbf{commutative}/\textbf{abelian} if
\begin{center}
    \(a * b = b * a\) for all \(a, b \in G\).
\end{center}

\prop. \note{Basic properties of groups} Let \(G\) be a group.
\begin{enumerate}
    \item \(G\) has a unique identity.
    \item For \(a \in G\), its inverse \(a\inv\) is unique.
    \item Left and right cancellation laws hold.
\end{enumerate}

\rmk \((G, *)\) is a group \miff \(*\) is associative, has left identity, has left inverse.

\pf \note{\mimpd} Let \(e\) be a left identity of \(G\). For any \(a \in G\), let \(a'\) be a left inverse of \(a\). Then, \(a' * a * e = e * e = e = a' * a = a' * e * a\). Let \(a''\) be a left inverse of \(a'\), then multiplying \(a''\) on the left gives
\[
    a'' * a' * a * e = a'' * a' * e * a \implies a * e = e * a = a,
\]
so \(a * e = a\), proving that \(e\) is also a right identity.\footnote{Alternatively, \(a e = e a e = (a'' a') a (a' a) = a'' e a' a = e a = a\).}

Let \(a'\) be a left inverse of \(a\), and let \(a''\) be a left inverse of \(a'\). Then \(a'' * a' * a * a' = e * a * a' = a * a'\), also \(a'' * a' * a * a' = a'' * e * a' = a'' * a' = e\). Therefore \(a * a' = e\), proving that \(a'\) is also a right inverse.\footnote{Alternatively, \(aa' = (a'' a')a a' = a'' (a' a) a' = a'' e a' = e\).}

\rmk For finite groups, the elements in a single row or column should be unique. For example, if some two elements \(a * x\), \(a * y\) in the same row (but different column) are the same, we can use the left cancellation law to show that \(x = y\). This is a contradiction.

So, let \(G = \{e, a\}\) be a group with identity \(e\). Then its operation table is determined uniquely, as the following.
\[
    \begin{array}{c|c|c}
        * & e & a \\ \hline
        e & e & a \\ \hline
        a & a & e
    \end{array}
\]
As for \(G = \{e, a, b\}\), it is also unique.
\[
    \begin{array}{c|c|c|c}
        * & e & a & b \\ \hline
        e & e & a & b \\ \hline
        a & a & b & e \\ \hline
        b & b & e & a
    \end{array}
\]

\pagebreak

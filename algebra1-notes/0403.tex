\section*{April 3rd, 2023}

\topic{Cosets and the Theorem of Lagrange}

\ex. Motivation.
\begin{enumerate}
    \item \(A_n \leq S_n\). We call \(S_n \bs A_n\) a coset of \(A_n\).
    \item \(3\Z \leq \Z\). We call \(3\Z\), \(3\Z + 1\), \(3\Z + 2\) are cosets of \(3\Z\).
\end{enumerate}

We saw that \(A_n\) and \(S_n \bs A_n\) have the same number of elements. This was done by constructing a bijection between two sets. But we know that a bijection between \(3\Z, 3\Z + 1, 3\Z + 2\) exist. We guess that the number of elements would be the same.

So if \(G\) is a finite group and \(H \leq G\), we conjecture that we can partition \(G\) by cosets of \(H\), and each cosets have the same number of elements. So \(\abs{H} \mid \abs{G}\).

\bigskip

\defn. Let \(G\) be a group, \(H \leq G\). Define a relation \(\sim_L\) on \(G\) as
\begin{center}
    \(a \sim_L b \iff a\inv b \in H\).
\end{center}

\rmk We can also define \(\sim_R\) on \(G\) as \(a \sim_R b \iff ba\inv \in H\).

\thm. \(\sim_L\) is an equivalence relation.

\pf
\begin{itemize}
    \item \(\forall a \in G\), \(a\inv a = e \in H\). \(a \sim_L a\).
    \item \(\forall a, b\in G\), if \(a\inv b \in H \implies (a\inv b)\inv = b\inv a \in H\). \(b \sim_L a\).
    \item\(\forall a, b, c\in G\), if \(a\inv b, b\inv c \in H \implies (a\inv b)(b\inv c) = a\inv c \in H\). \(a \sim_L c\).
\end{itemize}

\defn. \note{Coset} Let \(G, H\) be groups and \(H \leq G\). For \(a \in G\), we define
\begin{enumerate}
    \item The \textbf{left coset} of \(H\) as \(aH = \{ah : h \in H\}\).
    \item The \textbf{right coset} of \(H\) as \(Ha = \{ha : h \in H\}\).
\end{enumerate}

We see that \(a \neq b\) does not imply \(aH \neq bH\). If \(a, b\in H\), then \(aH = bH = H\). So when would we get the same coset?

\rmk\(aH = bH \iff H = a\inv b H \iff a \sim_L b\).

\ex.
\begin{enumerate}
    \item For a transposition \(\tau \in S_n\), \(\tau A_n\) is a coset of \(A_n\) with odd permutations. So it is different from \(A_n\). So for any odd permutation \(\sigma \in S_n\), \(\sigma A_n = \tau A_n\), and \(\sigma \sim_L \tau\).
    \item \(3\Z \leq \Z\). \(3\Z\), \(3\Z + 1\), \(3\Z + 2\) are cosets. Since \(\Z\) is commutative, the left and right cosets are equal to each other.
\end{enumerate}

\rmk If \(G\) is commutative and \(H \leq G\), \(aH = Ha\) for \(a \in G\).

Check for non-commutative groups that left and rights cosets need not be equal!

\medskip

\thm. \note{Lagrange} Let \(G\) be a finite group with \(H \leq G\). Then \(\abs{H} \mid \abs{G}\).

\pf If we construct a bijection between any two left cosets, \(\abs{H}\) would be \(\abs{G}\) divided by the number of left cosets. Which would imply \(\abs{H} \mid \abs{G}\).

Recall that left cosets are defined as the equivalence classes with respect to the relation \(\sim_L\). So \(G\) is a disjoint union of cosets, \(G = \bigsqcup aH\). Therefore, \(\abs{G} = \text{(number of left cosets)} \cdot \abs{H}\).

\lemma. \(\abs{aH} = \abs{bH}\) for \(a \in G\), \(H \leq G\).

\pf \(\varphi : aH \ra bH\) is a bijection. Check by yourself!

\cor. The number of left cosets and the number of right cosets are equal.

\medskip

\cor. Every group of prime order is cyclic.

\pf Let \(\abs{G} = p\), where \(p\) is prime. Take \(a \in G\) which is not the identity. Then the cyclic subgroup \(\span{a} \leq G\) must have order 1 or \(p\). But \(a\) must have order \(p\) because it is not the identity. So \(\abs{\span{a}} = p\), which implies that \(G = \span{a}\).

This is a very important result related to the classification of finite (simple) groups. We have seen all groups of order 2, 3, 4. But for larger orders, we can't enumerate them all. With this result, we directly know that for groups with prime order \(p\), the group is isomorphic to \(\Z_p\).

\medskip

This is also a direct result of Lagrange's theorem, since \(\span{a} \leq G\).

\thm. The order of an element in a finite group divides the order of the group. In other words, \(\abs{\span{a}} \mid \abs{G}\), for \(a \in G\).

\defn. \note{Index} Let \(G\) be a group, (not necessarily finite) and \(H \leq G\). We define the \textbf{index} of \(H\) in \(G\) as
\[
    \ind{G}{H} = \text{number of left cosets of } H = \text{number of right cosets of } H.
\]
If \(\abs{G} < \infty\), \(\ind{G}{H} = \abs{G} / \abs{H}\).

\thm. For a group \(G\), suppose that \(K \leq H \leq G\). If \(\ind{G}{H}\), \(\ind{H}{K}\) are finite,
\[
    \ind{G}{K} = \ind{G}{H}\ind{H}{K}.
\]
\pf Write \(G = \bigsqcup_{i=1}^n a_i H\), \(H = \bigsqcup_{j=1}^m b_j K\), and show that \(G = \bigsqcup_{i, j} a_i b_j K\). Check by yourself!

\pagebreak

\topic{Direct Products, Finitely Generated Abelian Groups}

\defn. \note{Cartesian Product} For sets \(S_1, \dots, S_n\), define
\[
    \prod_{i=1}^n S_i = S_1 \times S_2 \times \cdots \times S_n = \{(a_1, \dots, a_n) : a_i \in S_i\}.
\]

What if \(S_i\) already have a group structure?

\defn. \note{Direct Product} Let \(G_1, \dots, G_n\) be groups. Define a binary operation \(\cdot\) as
\[
    (a_1, a_2, \dots, a_n) \cdot (b_1, b_2, \dots, b_n) = (a_1b_1, a_2b_2, \dots, a_nb_n).
\]
Then the \textbf{direct product} \(\prod_{i=1}^n G_i\) is a group with this binary operation.

\notation We also write the direct sum as \(\bigoplus_{i=1}^n G_i\) for additive groups.

\ex. Compare the Klein 4-group \(V_4\) with \(\Z_2 \times \Z_2\). They have the exact same structure! \(V_4 \simeq \Z_2 \times \Z_2\). From this example, we found out that order 4 group is either \(\Z_4\) or \(\Z_2 \times \Z_2\).

Now we know all groups of order up to 5. How about order 6? We know \(S_3\) and \(\Z_6\).

\ex. Consider \(\Z_2 \times \Z_3\). We can check that \((1, 1) \in \Z_2 \times \Z_3\) has order 6, so \(\span{(1, 1)} = \Z_2 \times \Z_3 \simeq \Z_6\).

Why was it that \(\Z_4 \neq \Z_2 \times \Z_2\), but \(\Z_6 \simeq \Z_2 \times \Z_3\)?

\thm. \(\Z_m \times \Z_n \simeq \Z_{mn}\) if and only if \(\gcd(m, n) = 1\).

\pf \\
\note{\mimpd} We need to find a generator of \(\Z_m \times \Z_n\) with order \(mn\). Take \(a = (1, 1) \in \Z_m \times \Z_n\). The order of this element should be divisible by \(m, n\). So \(mn\) is the smallest positive integer, and \(\abs{\span{(1, 1)}} = mn\).

\note{\mimp} Suppose that \(d = \gcd(a, b) > 1\). Then for all \((a, b) \in \Z_m \times \Z_n\), \(\frac{mn}{d}(a, b) = (0, 0)\). So \((a, b)\) cannot generate the entire group (which has \(mn\) elements). Therefore \(\Z_m \times \Z_n\) is not cyclic.

\pagebreak

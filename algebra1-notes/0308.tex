\chapter{Binary Operations}

\section*{March 8th, 2023}

\defn. \note{Binary Operation} A \textbf{binary operation} \(\ast\) on a set \(S\) is defined as a function
\[
    \ast: S \times S \ra S
\]
We write \(a \ast b\) instead of \(\ast(a, b)\).

\rmk If you consider the number systems like \(\N, \Z, \Q, \R, \C\), we already know operations defined on them. We are just naming them \textit{binary operations}. The examples came first, and the definitions came afterwards. We can also see that the definitions are really useful, and it will serve as a tool for formalizing our theory.

\ex. Examples of binary operations.
\begin{enumerate}
    \item Addition \(+\) and multiplication \(\cdot\) on \(\N, \Z, \Q, \R, \C\).
    \item Subtraction \(-\) is not a binary operation on \(\N\). We extend \(\N\) to \(\Z\), so that \(-\) is a binary operation on \(\Z\).
    \item The set of functions \(f: \R \ra \R\), and \(+, -, \times, \circ\) defined on it.
\end{enumerate}

\defn. \note{Closure} For a set \(S\) and a binary operation \(\ast\) on \(S\), suppose that \(H \subset S\). We restrict the domain of \(\ast\) to \(H \times H\). Then
\[
    \ast\mid_{H\times H} \;: H\times H \ra S.
\]
If the image of \(\ast\mid_{H\times H} \;\subset H\), then we say that \textbf{\(H\) is closed under \(\ast\)}.

\rmk When we learn new definitions, it's very important to think about examples that we already know. Often books give trivial examples first, and then show some non-trivial examples, motivating us to study. Books are written in that way!

\defn. Let \((S, \ast)\) be given. We say that
\begin{enumerate}
    \item \(\ast\) is \textbf{commutative} if \(a \ast b = b \ast a\) for all \(a, b \in S\).
    \item \(\ast\) is \textbf{associative} if \((a \ast b)\ast c = a \ast (b \ast c)\) for all \(a, b, c \in S\).
\end{enumerate}

\ex. Consider these binary operations on \(\Z\).
\begin{enumerate}
    \item \(a \ast b = a\), \(\ast\) is not commutative but associative.
    \item \(a \ast b = a + 2\), \(\ast\) is not commutative nor associative.
\end{enumerate}

\rmk To study these binary operations, we first start with finite sets where we can write tables with the results of the binary operation.
\[
    S = \{a\} \implies \begin{array}{c|c}
        \ast & a \\ \hline
        a    & a
    \end{array}\;,
    \qquad
    S = \{a, b\} \implies
    \begin{array}{c|c|c}
        \ast & a        & b     \\ \hline
        a    &          & a * b \\ \hline
        b    & b \ast a &
    \end{array}
\]
Consider the relation between binary operation on a finite set \(S\) and tables. Binary operation is a function, so we have the existence and uniqueness of \(a \ast b\). In terms of tables, we see that each cell in the table should have a value and it should be uniquely determined. So we conclude that \textit{we can describe a binary operation with a table}.

\pagebreak
